\documentclass[12pt,a4paper]{article}
\usepackage[utf8]{inputenc}
\usepackage[english]{babel}

\usepackage{amsmath}
\usepackage{amsfonts}
\usepackage{amssymb}

\usepackage{graphicx}
\usepackage{lmodern}
\usepackage{tikz}
\usepackage{titlesec}
\usepackage{environ}
\usepackage{xcolor}
\usepackage{fancyhdr}
\usepackage[colorlinks = true, linkcolor = black]{hyperref}
\usepackage{xparse}
\usepackage{enumerate}

\usepackage[left=2cm,right=2cm,top=2cm,bottom=2cm]{geometry}
\usepackage{multicol}

\newcommand{\spaceP}{\vspace*{0.5cm}}

%% Redefining sections
\newcommand{\sectionformat}[1]{%
    \begin{tikzpicture}[baseline=(title.base)]
        \node[rectangle, draw] (title) {#1};
    \end{tikzpicture}
    
    \noindent\hrulefill
}

% default values copied from titlesec documentation page 23
% parameters of \titleformat command are explained on page 4
\titleformat%
    {\section}% <command> is the sectioning command to be redefined, i. e., \part, \chapter, \section, \subsection, \subsubsection, \paragraph or \subparagraph.
    {\normalfont\large\scshape}% <format>
    {}% <label> the number
    {0em}% <sep> length. horizontal separation between label and title body
    {\centering\sectionformat}% code preceding the title body  (title body is taken as argument)

%% Set counters for sections to none
\setcounter{secnumdepth}{0}

%% Set the footer/headers
\pagestyle{fancy}
\fancyhf{}
\renewcommand{\headrulewidth}{0pt}
\renewcommand{\footrulewidth}{2pt}
\lfoot{P.-O. Paris{\'e}}
\cfoot{MATH 307}
\rfoot{Page \thepage}

%% Defining example environment
\newcounter{example}[section]
\NewEnviron{example}%
	{%
	\noindent\refstepcounter{example}\fcolorbox{gray!40}{gray!40}{\textsc{\textcolor{red}{Example~\theexample.}}}%
	%\fcolorbox{black}{white}%
		{  %\parbox{0.95\textwidth}%
			{
			\BODY
			}%
		}%
	}

% Theorem environment
\NewEnviron{theorem}%
	{%
	\noindent\refstepcounter{example}\fcolorbox{gray!40}{gray!40}{\textsc{\textcolor{blue}{Theorem~\theexample.}}}%
	%\fcolorbox{black}{white}%
		{  %\parbox{0.95\textwidth}%
			{
			\BODY
			}%
		}%
	}

%% Commands for matrix of dimensions m x n
%\newcommand{\matMN}[1]{%
%	\begin{bmatrix}
%	#1_{11} & #1_{12} & #1_{13} & \cdots & #1_{1n} \\
%	#1_{21} & #1_{22} & #1_{23} & \cdots & #1_{2n} \\
%	#1_{31} & #1_{32} & #1_{33} & \cdots & #1_{3n} \\
%	\vdots & \vdots & \vdots & \ddots & \vdots \\
%	#1_{m1} & #1_{m2} & #1_{m3} & \cdots & #1_{mn}
%	\end{bmatrix}		
%	}
	
\NewDocumentCommand{\matMN}{mg}{%
	\IfNoValueTF{#2}
		{%
		  \begin{bmatrix}
		   #1_{11} & #1_{12} & #1_{13} & \cdots & #1_{1n} \\
	       #1_{21} & #1_{22} & #1_{23} & \cdots & #1_{2n} \\
		   #1_{31} & #1_{32} & #1_{33} & \cdots & #1_{3n} \\
		   \vdots & \vdots & \vdots & \ddots & \vdots \\
		   #1_{m1} & #1_{m2} & #1_{m3} & \cdots & #1_{mn}
		   \end{bmatrix}	%
		}
		{%
		   \begin{bmatrix}
		   #1_{11} + #2_{11} & #1_{12} + #2_{12} & \cdots & #1_{1n} + #2_{2n} \\
		   #1_{21} + #2_{21} & #1_{22} + #2_{22} & \cdots & #1_{2n} + #2_{2n} \\
		   \vdots & \vdots & \vdots & \ddots & \vdots \\
		   #1_{m1} + #2_{m1} & #1_{m2} + #2_{m2} & \cdots & #1_{mn} + #2_{mn}
		   \end{bmatrix}
		}%
}

%%%%

\begin{document}
\thispagestyle{empty}

\begin{center}
\vspace*{2.5cm}

{\Huge \textsc{Math 307}}

\vspace*{2cm}

{\LARGE \textsc{Chapter 1}} 

\vspace*{0.75cm}

\noindent\textsc{Section 1.2: Matrices and Matrix Operations}

\vspace*{0.75cm}

\tableofcontents

\vfill

\noindent \textsc{Created by: Pierre-Olivier Paris{\'e}} \\
\textsc{Summer 2022}
\end{center}

\newpage

\section{What are Matrices?}

\subsection{Definition}
A matrix is a bunch of numbers arranged in $m$ rows and $n$ columns like an array:
	\begin{align*}
	A = \begin{bmatrix}
	a_{11} & a_{12} & a_{13} & \cdots & a_{1n} \\
	a_{21} & a_{22} & a_{23} & \cdots & a_{2n} \\
	a_{31} & a_{32} & a_{33} & \cdots & a_{3n} \\
	\vdots & \vdots & \vdots & \ddots & \vdots \\
	a_{m1} & a_{m2} & a_{m3} & \cdots & a_{mn}
	\end{bmatrix} .
	\end{align*}

\begin{multicols}{2}
\begin{itemize}
\item $a_{ij}$ are the elements/entries of the matrix. Here are some notations to describe the entries of a matrix:
	\begin{itemize}
	\item $\mathrm{ent}_{ij} (A)$.
	\item $A = [a_{ij}]$.
	\end{itemize}
\item The dimensions of a matrix is the number of rows ($m$) and of columns ($n$). To specify the dimensions, we say \textit{an $m \times n$ matrix $A$}.
\item The set of all matrices of dimensions $m \times n$ is denoted by $M_{m \times n} (\mathbb{R})$.
\end{itemize}
\end{multicols}

\vspace*{12pt}

\begin{example}
Here are some examples of matrices:
	\begin{align*}
	A = \begin{bmatrix}
	1 & 2 & 3 \\
	4 & 5 & 6 
	\end{bmatrix} ,
	\quad
	B = \begin{bmatrix}
	1 & 2 & 3 \\
	4 & 5 & 6 \\
	7 & 8 & 9
	\end{bmatrix} ,
	\quad
	C = \begin{bmatrix}
	1 \\ -2 \\ 4 \\ 0 
	\end{bmatrix} , 
	\quad
	D = \begin{bmatrix}
	-1 & 3 & 0 & 0.5 & \pi 
	\end{bmatrix} .
	\end{align*}
\end{example}

\subsection{Some Types of Matrices}

\begin{itemize}
\item A row matrix is a matrix a dimensions $1 \times n$.
	\begin{itemize}
	\item \underline{Remark}: row matrices are models for row vectors.
	\end{itemize}
\item A column matrix is a matrix of dimensions $m \times 1$. 
	\begin{itemize}
	\item \underline{Remark}: column matrices are models for column vectors.
	\end{itemize}
\item Square matrices are matrices of dimensions $n \times n$ (same number of rows and number of columns). 
	\begin{itemize}
	\item The elements $a_{11}$, $a_{22}$, $\ldots$, $a_{nn}$ of a square matrix are called diagonal entries.
	\end{itemize}
\end{itemize}

\begin{example}
For each matrix in Example 1,
	\begin{multicols}{2}
	\begin{enumerate}
	\item Give the dimensions;
	\item If possible, give the type of the matrix.
	\end{enumerate}
	\end{multicols}
\end{example}

\newpage

\section{Arithmetic with Matrices}

\subsection{Equality}
Two matrices $A = [a_{ij}]$ and $B = [b_{ij}]$ are equal, written $A = B$, if 
	\begin{itemize}
	\item they have the same dimensions and;
	\item they have the same entries, that is $a_{ij} = b_{ij}$ for all indices $i$, $j$.
	\end{itemize}
	
	\begin{example}
	Determine if the matrices $A$ and $B$ are equal.
		\begin{multicols}{2}
		\begin{enumerate}
		\item $A = \begin{bmatrix}
		-1 & 2 \\ 1 & 12
		\end{bmatrix}$ and $B = \begin{bmatrix}
		5 & 2 \\ 1 & 12
		\end{bmatrix}$.
		\item $A = \begin{bmatrix}
		1 \\ 2
		\end{bmatrix}$ and $B = \begin{bmatrix}
		1 & 2
		\end{bmatrix}$.
		\end{enumerate}
		\end{multicols}
	\end{example}
	
	\vspace*{2cm}

\subsection{Addition}
A matrix $A = [a_{ij}]_{m \times n}$ is added to another matrix $B = [b_{ij}]_{m \times n}$ in the following way:

\begin{footnotesize}
	\begin{align*}
	\matMN{a} + \matMN{b} := \matMN{a}{b}
	\end{align*}
\end{footnotesize}

\vspace*{8pt}

\noindent\underline{Remark:} To add two matrices together, they must have the same dimensions! Otherwise, it doesn't make sense!

\vspace*{16pt}
	
	\begin{example}
	Add together the given matrices, if possible.
	\begin{multicols}{2}
		\begin{enumerate}[a)]
		\item $A = \begin{bmatrix}
		1 & 2 \\ 3 & 4 \\ 5 & 6
		\end{bmatrix}$ and
		$B = \begin{bmatrix}
		8 & 9 \\
		10 & 11 \\
		12 & 13
		\end{bmatrix}$.
		\item $A = \begin{bmatrix}
		-2 & 5 \\ 3 & 1
		\end{bmatrix}$ and
		$B = \begin{bmatrix}
		3 & 0 & -2
		\end{bmatrix}$.
		\end{enumerate}
	\end{multicols}
	\end{example}

\newpage

\subsection{Scalar Multiplication}
We can multiply a matrix by a number. Given a number $t$, we define 
	\begin{align*}
	t \matMN{a} := \matMN{t a} .
	\end{align*}

\noindent\underline{Remark}: A number will be called, in many circumstances, a scalar.

\vspace*{16pt}
	
	\begin{example}
	Give the resulting matrix after performing the following operations:
		\begin{align*}
		2 \begin{bmatrix}
		1 & 2 \\ 3 & 4
		\end{bmatrix} - \begin{bmatrix}
		-1 & -2 \\ 1 & 2
		\end{bmatrix} .
		\end{align*}
	\end{example}
	
	\vfill
	
	\noindent\underline{Remarks}:
		\begin{itemize}
		\item In general, the operation $A - B$ is defined by $A + (-B)$ where $-B$ is $B$ multiplied by the scalar $-1$.
		\item The zero matrix $O_{m \times n}$ of dimensions $m \times n$ is the matrix containing only zeros:
			\begin{align*}
			O_{2 \times 3} = \begin{bmatrix}
			0 & 0 & 0 \\ 0 & 0 & 0
			\end{bmatrix}
			\quad O_{3 \times 1} = \begin{bmatrix}
			0 \\ 0 \\ 0
			\end{bmatrix} .
			\end{align*}
		\item Adding the zero matrix doesn't change anything: $O_{m \times n} + A = A + O_{m \times n} = A$.
		\item Multiplying by the number $0$ changes all the entries of the matrix $A$: $0 A = O_{m \times n}$.
		\item Here are basic arithmetic with the addition and scalar multiplication: If $A$, $B$, and $C$ are matrices of the same size and if $s$ and $t$ are numbers, then
		\begin{multicols}{2}
			\begin{enumerate}
			\item $A + B = B + A$.
			\item $A + (B + C) = (A + B) + C$ .
			\item $s (tA) = (st) A$.
			\item $s (A + B) = sA + sB$.
			\item $(s + t) A = sA + tA$.
			\end{enumerate}
		\end{multicols}
	\end{itemize}

\newpage

\subsection{Matrix Multiplication}

	\noindent\underline{\textbf{Row Matrix times a Column Vector}}
	
	\vspace*{12pt}
	
	\begin{example}
	In a hale'kū'ai, there are selling oranges, pineaples, and mangos. A pound of oranges cost \$$2$, a pound of pineaples is \$$3$, and a pound of mangos is \$$4$. You buy $2$ pounds of oranges, $3$ pounds of pineaples, and $2$ pounds of mangos. What is the total cost of your purchase?
	\end{example}
	
	\vfill
	
	\noindent\underline{In General}: Given a row vector $v = \begin{bmatrix} a_1 & a_2 & \cdots & a_n \end{bmatrix}$ and a column vector
		\begin{align*}
		u = \begin{bmatrix}
		b_1 \\ b_2 \\ \vdots \\ b_n
		\end{bmatrix} ,
		\end{align*}
	we define the product $v u$ by
		\begin{align*}
		v u := a_1 b_1 + a_2 b_2 + \cdots + a_n b_n .
		\end{align*}
		
	\noindent\underline{Remark}: 
		\begin{itemize}
		\item We see that multiplying a matrix of dimensions $1 \times n$ with a matrix of dimensions $n \times 1$ gives a matrix of dimensions $1 \times 1$ (a number).
		\item If the number of elements in the vector $v$ would be different from the number of elements in the vector $u$, then the product $vu$ doesn't make sense. In this case, we can't perform the operation!
		\end{itemize}
		

\newpage

	\noindent\underline{\textbf{Matrix times a Column Vector}}
	
	\vspace*{12pt}
	
	\begin{example}
	Your friend and you are in the same hale'kū'ai. You decide to buy $5$ pounds of oranges, $2$ pounds of pineaples and $1$ pound mangos. Your friend, prefering mangos over the other fruits, buys $2$ pounds of oranges, $2$ pounds of pineaples, and $10$ pounds of mangos. How much did it cost to you and your friend?
	\end{example}
	
	\vfill
	
	\noindent\underline{In General}: Given a $m \times n$ matrix $A$ and a $n \times 1$ column vector $u$, the product $Au$ is the new matrix $C = [c_j]_{1 \leq j \leq m}$ of dimensions $m \times 1$ where
		\begin{align*}
		c_1 &= a_{11} b_1 + a_{12} b_2 + \cdots + a_{1n} b_n \\
		c_2 &= a_{21} b_1 + a_{22} b_2 + \cdots + a_{2n} b_n \\
		& \qquad \vdots \\
		c_m &= a_{m1} b_1 + a_{m2} b_2 + \cdots + a_{mn} b_n .
		\end{align*}
		
	\noindent\underline{Remarks}:
		\begin{itemize}
		\item Multiplying a matrix of dimensions $m \times n$ with a matrix of dimensions $n \times 1$ results in a matrix of dimensions $m \times 1$.
		\item To multiplying a matrix with a column vector, the number of columns in the matrix must be the same as the number of elements in the column vector.
		\end{itemize}
	
\newpage

	\noindent\underline{\textbf{Matrix times a Matrix}}
	
	\vspace*{12pt}
	
	\noindent Let $A$ and $B$ be two matrices
		\begin{align*}
		A = \begin{bmatrix}
		a_{11} & a_{12} & a_{13} & \cdots & a_{1n} \\
		a_{21} & a_{22} & a_{23} & \cdots & a_{2n} \\
		a_{31} & a_{32} & a_{33} & \cdots & a_{3n} \\
		\vdots & \vdots & \vdots & \ddots & \vdots \\
		a_{m1} & a_{m2} & a_{m3} & \cdots & a_{mn}
		\end{bmatrix} 
		\quad \text{ and } \quad
		B = \begin{bmatrix}
		b_{11} & b_{12} & b_{13} & \cdots & b_{1k} \\
		b_{21} & b_{22} & b_{23} & \cdots & b_{2k} \\
		b_{31} & b_{32} & b_{33} & \cdots & b_{3k} \\
		\vdots & \vdots & \vdots & \ddots & \vdots \\
		b_{n1} & b_{n2} & b_{n3} & \cdots & b_{nk}
		\end{bmatrix} .
		\end{align*}
		
	\noindent We will adopt the following notation
		\begin{align*}
		B = \begin{bmatrix}
		B_1 & B_2 & B_3 & \cdots & B_k
		\end{bmatrix}
		\end{align*}
	where $B_1$, $B_2$, $\ldots$, $B_k$ are the columns of $B$:
		\begin{align*}
		B_1 = \begin{bmatrix}
		b_{11} \\ b_{21} \\ \vdots \\ b_{m1}
		\end{bmatrix} , \quad
		B_2 = \begin{bmatrix}
		b_{21} \\ b_{22} \\ \vdots \\ b_{m2}
		\end{bmatrix} , \quad \cdots , \quad
		B_k = \begin{bmatrix}
		b_{1k} \\ b_{2k} \\ \vdots \\ b_{mk}
		\end{bmatrix} .
		\end{align*}
	
	
	\vspace*{16pt}
	
	\noindent The multiplication of $A$ with $B$ is defined as the new matrix $C$ of dimensions $m \times k$:
		\begin{align*}
		C = \begin{bmatrix}
		A B_1 & A B_2 & A B_3 & \cdots & A B_k
		\end{bmatrix} .
		\end{align*}
	In other words, the columns $C_1$, $C_2$, $\ldots$, $C_k$ of the matrix $C$ is the column vectors $AB_1$, $AB_2$, $\ldots$, $AB_k$.
	
	\vspace*{16pt}
	
	\noindent\underline{Remarks}:
	\begin{itemize}
	\item To multiply a matrix $A$ with a matrix $B$, the number of columns of $A$ must agree with the number of rows of $B$. If this is not satisfied, the product $AB$ is not defined!
	\item The identity matrix is the square matrix $I_n$ with all $1$ on the diagonal and zeros elsewhere:
		\begin{align*}
		I_2 = \begin{bmatrix}
		1 & 0 \\ 0 & 1
		\end{bmatrix} , \quad
		I_3 = \begin{bmatrix}
		1 & 0 & 0 \\
		0 & 1 & 0 \\
		0 & 0 & 1
		\end{bmatrix} \quad \text{ and } \quad
		I_4 = \begin{bmatrix}
		1 & 0 & 0 & 0 \\
		0 & 1 & 0 & 0 \\
		0 & 0 & 1 & 0 \\
		0 & 0 & 0 & 1
		\end{bmatrix} .
		\end{align*}
	\item The identity matrix is such that $I_m A = A$ and $A I_n = A$ for any matrix $A$ of dimensions $m \times n$.
	\item We can multiply several times the same matrix with itself. This is the \textit{powers of a matrix}. If $A$ is a square matrix, then
		\begin{align*}
		A^1 = A , \quad A^2 = AA , \quad A^3 = A^2 A = AAA , \cdots \quad A^n = A^{n-1} A = \underbrace{AA \cdots A}_{n \text{ times}}
		\end{align*}
	\item Here are some basic arithmetic with the matrix multiplication:
		\begin{enumerate}
		\item $A (BC) = (AB)C$;
		\item $A (B + C) = AB + AC$;
		\item $(A + B)C = AC + BC$;
		\item $\lambda (AB) = (\lambda A) B = A (\lambda B)$.
		\end{enumerate}
	\end{itemize}
	
\newpage
	
	\begin{example}
	If possible, compute the products $AB$ and $BA$ where
		\begin{align*}
		A = \begin{bmatrix}
		1 & 2 & -1 \\
		1 & -3 & 1 \\
		-2 & 2 & 1
		\end{bmatrix}
		\quad \text{and} \quad
		B = \begin{bmatrix}
		0 & 0 \\
		1 & 0 \\
		0 & 1 
		\end{bmatrix} .
		\end{align*}
	\end{example}
	
\newpage

	\begin{example}
	If possible, compute the products $AB$ and $BA$ where
		\begin{align*}
		A = \begin{bmatrix}
		1 & 2 & 3 \\
		2 & 4 & 6 \\
		-1 & -2 & -3
		\end{bmatrix}
		\quad \text{and} \quad
		B = \begin{bmatrix}
		0 & 0 & 1 \\
		1 & 0 & 0 \\
		0 & 1 & 0
		\end{bmatrix} .
		\end{align*}
	\end{example}
	
	\vfill
	
	\noindent\underline{Remark}:
	
	\vspace*{2cm}
	
\newpage
	
\section{Connection with System of Linear Equations}
Consider
	\begin{align*}
	a_{11} x_1 + a_{12} x_2 + \cdots + a_{1n} x_n = b_1 \\
	a_{21} x_1 + a_{22} x_2 + \cdots + a_{2n} x_n = b_2 \\
	\vdots \phantom{aaaaaaaaaaaaa} \\
	a_{m1} x_1 + a_{m2} x_2 + \cdots + a_{mn} x_n = b_m  .
	\end{align*}
	
\subsection{Augmented Matrix Notation}
	
	\begin{align*}
	\begin{bmatrix}
	a_{11} & a_{12} & \cdots & a_{1n} & b_1 \\
	a_{21} & a_{22} & \cdots & a_{2n} & b_2 \\
	\vdots & \vdots & \vdots & \vdots & \vdots \\
	a_{m1} & a_{m2} & \cdots & a_{mn} & b_m
	\end{bmatrix}
	\end{align*}
	
One side of the coin!

\subsection{Rewriting a System in Matrix form}
Letting
	\begin{align*}
	A = \matMN{a} , \quad X = \begin{bmatrix}
	x_1 \\ x_2 \\ x_3 \\ \vdots \\ x_n
	\end{bmatrix}
	\quad \text{ and } \quad B = \begin{bmatrix}
	b_1 \\ b_2 \\ b_3 \\ \vdots \\ b_m 
	\end{bmatrix}
	\end{align*}
we can rewrite the system as	
	\begin{align*}
	AX = B	.
	\end{align*}
	
\begin{example}
Rewrite the following system in its matrix form:
	\begin{align*}
	3x_1 + 4x_2 - 5x_3 = 1 \\
	5x_1 + 5x_2 - 3x_3 = 2 \\
	-2 x_1 - 5x_2 + 0.5 x_3 = 3 .
	\end{align*}
\end{example}


\end{document}
