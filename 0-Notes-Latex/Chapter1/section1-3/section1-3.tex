\documentclass[12pt,a4paper]{article}
\usepackage[utf8]{inputenc}
\usepackage[english]{babel}

\usepackage{amsmath}
\usepackage{amsfonts}
\usepackage{amssymb}

\usepackage{graphicx}
\usepackage{lmodern}
\usepackage{tikz}
\usepackage{titlesec}
\usepackage{environ}
\usepackage{xcolor}
\usepackage{fancyhdr}
\usepackage[colorlinks = true, linkcolor = black]{hyperref}
\usepackage{xparse}
\usepackage{enumerate}
\usepackage{parskip}

\usepackage[left=2cm,right=2cm,top=2cm,bottom=2cm]{geometry}
\usepackage{multicol}

\newcommand{\spaceP}{\vspace*{0.5cm}}

%% Redefining sections
\newcommand{\sectionformat}[1]{%
    \begin{tikzpicture}[baseline=(title.base)]
        \node[rectangle, draw] (title) {#1};
    \end{tikzpicture}
    
    \noindent\hrulefill
}

% default values copied from titlesec documentation page 23
% parameters of \titleformat command are explained on page 4
\titleformat%
    {\section}% <command> is the sectioning command to be redefined, i. e., \part, \chapter, \section, \subsection, \subsubsection, \paragraph or \subparagraph.
    {\normalfont\large\scshape}% <format>
    {}% <label> the number
    {0em}% <sep> length. horizontal separation between label and title body
    {\centering\sectionformat}% code preceding the title body  (title body is taken as argument)

%% Set counters for sections to none
\setcounter{secnumdepth}{0}

%% Set the footer/headers
\pagestyle{fancy}
\fancyhf{}
\renewcommand{\headrulewidth}{0pt}
\renewcommand{\footrulewidth}{2pt}
\lfoot{P.-O. Paris{\'e}}
\cfoot{MATH 307}
\rfoot{Page \thepage}

%% Defining example environment
\newcounter{example}[section]
\NewEnviron{example}%
	{%
	\noindent\refstepcounter{example}\fcolorbox{gray!40}{gray!40}{\textsc{\textcolor{red}{Example~\theexample.}}}%
	%\fcolorbox{black}{white}%
		{  %\parbox{0.95\textwidth}%
			{
			\BODY
			}%
		}%
	}

% Theorem environment
\NewEnviron{theorem}%
	{%
	\noindent\refstepcounter{example}\fcolorbox{gray!40}{gray!40}{\textsc{\textcolor{blue}{Theorem~\theexample.}}}%
	%\fcolorbox{black}{white}%
		{  %\parbox{0.95\textwidth}%
			{
			\BODY
			}%
		}%
	}

%% Commands for matrix of dimensions m x n
%\newcommand{\matMN}[1]{%
%	\begin{bmatrix}
%	#1_{11} & #1_{12} & #1_{13} & \cdots & #1_{1n} \\
%	#1_{21} & #1_{22} & #1_{23} & \cdots & #1_{2n} \\
%	#1_{31} & #1_{32} & #1_{33} & \cdots & #1_{3n} \\
%	\vdots & \vdots & \vdots & \ddots & \vdots \\
%	#1_{m1} & #1_{m2} & #1_{m3} & \cdots & #1_{mn}
%	\end{bmatrix}		
%	}
	
\NewDocumentCommand{\matMN}{mg}{%
	\IfNoValueTF{#2}
		{%
		  \begin{bmatrix}
		   #1_{11} & #1_{12} & #1_{13} & \cdots & #1_{1n} \\
	       #1_{21} & #1_{22} & #1_{23} & \cdots & #1_{2n} \\
		   #1_{31} & #1_{32} & #1_{33} & \cdots & #1_{3n} \\
		   \vdots & \vdots & \vdots & \ddots & \vdots \\
		   #1_{m1} & #1_{m2} & #1_{m3} & \cdots & #1_{mn}
		   \end{bmatrix}	%
		}
		{%
		   \begin{bmatrix}
		   #1_{11} + #2_{11} & #1_{12} + #2_{12} & \cdots & #1_{1n} + #2_{2n} \\
		   #1_{21} + #2_{21} & #1_{22} + #2_{22} & \cdots & #1_{2n} + #2_{2n} \\
		   \vdots & \vdots & \vdots & \ddots & \vdots \\
		   #1_{m1} + #2_{m1} & #1_{m2} + #2_{m2} & \cdots & #1_{mn} + #2_{mn}
		   \end{bmatrix}
		}%
}

%%%%
\begin{document}
\thispagestyle{empty}

\begin{center}
\vspace*{2.5cm}

{\Huge \textsc{Math 307}}

\vspace*{2cm}

{\LARGE \textsc{Chapter 1}} 

\vspace*{0.75cm}

\noindent\textsc{Section 1.3: Inverses of Matrices}

\vspace*{0.75cm}

\tableofcontents

\vfill

\noindent \textsc{Created by: Pierre-Olivier Paris{\'e}} \\
\textsc{Summer 2022}
\end{center}

\newpage

\setlength{\parindent}{0pt}

\section{What is an Inverse?}

\subsection{For Real Numbers}
\begin{example}
Find the value of $x$ if
	\begin{enumerate}
	\item $2x - 1 = 0$.
	\item $x^2 - x = 0$.
	\end{enumerate}
\end{example}

\vspace*{3cm}

\noindent\underline{Secretly}:
	\begin{itemize}
	\item In the first equation, we multiplied by the inverse of $2$, which is $1/2$, because $(1/2) 2 = 1$.
	\item In the second equation, we examined the values of $x$ and made sure we avoid the value $0$ because $0$ is not "divisible". In other words, it doesn't have an inverse.
	\end{itemize}

\subsection{For Matrices}
We say that a square matrix $A$ is invertible if there is another matrix $B$ such that
	\begin{align*}
	A B = BA = I .
	\end{align*}
	
\noindent\underline{Remarks}:
		\begin{enumerate}[\textbullet]
		\item Not all non-zero square matrices are invertible.
		\item Matrices that are invertible are called \textbf{nonsingular} and matrices that are not invertible are called \textbf{singular}.
		\item If the inverse exists, then there is only one inverse and we denote it by $A^{-1}$.
		\end{enumerate}
	
\begin{example}
Verify that the matrix $B$ is the inverse of $A$ if
	\begin{align*}
	A = \begin{bmatrix}
	1 & 2 \\ 3 & 5
	\end{bmatrix}
	\quad
	B = \begin{bmatrix}
	-5 & 2 \\ 3 & -1
	\end{bmatrix} .
	\end{align*}
\end{example}

\newpage

\subsection{Properties of Inverses}

\begin{example}
Find the inverse of the product
	\begin{align*}
	\begin{bmatrix}
	1 & 0 & 0 \\ 0 & 1 & 2 \\ 0 & 0 & 1
	\end{bmatrix} 
	\begin{bmatrix}
	2 & 1 & 3 \\ 2 & 1 & 1 \\ 4 & 5 & 1
	\end{bmatrix} .
	\end{align*}
\end{example}

\vfill

\noindent \underline{\textsc{General Facts}}: Let $A$ and $B$ be matrices of the same size and let $m$ be a positive integer.
	\begin{itemize}
	\item If $A$ and $B$ are invertible, then $AB$ is invertible with $(AB)^{-1} = B^{-1} A^{-1}$.
	\item If $A$ is invertible, then $A^{-1}$ is also invertible and $(A^{-1})^{-1} = A$.
	\item If $A$ is invertible, then $A^m$ is also invertible and $(A^m)^{-1} = (A^{-1})^m$.
	\end{itemize}
	\begin{itemize}
	\item Suppose that $A$ and $B$ are $n \times n$ matrices such that $AB = I$ or $BA = I$. Then $A$ has an inverse and $A^{-1} = B$.
	\end{itemize}

\newpage

\section{How do we find the inverse?}

For numbers, finding the inverses is quite straightforward, or should we say "we are used to divide with numbers". 

\subsection{Little Warm-up}
For matrices, it is not that obvious.

\vspace*{16pt}

	\begin{example}
	Find the inverse of the matrix
		\begin{align*}
		A = \begin{bmatrix}
		1 & 2 \\ 3 & 5
		\end{bmatrix} .
		\end{align*}
	\end{example}
	
\newpage
	
\subsection{Systematic method with Augmented Matrices}
Given a square matrix $A = [a_{ij}]$, we "augment" $A$ with the identity matrix:
	\begin{align*}
	\begin{bmatrix}
	A & I
	\end{bmatrix}
	=
	\begin{bmatrix}
	a_{11} & a_{12} & \cdots & a_{1n} & 1 & 0 & \cdots & 0 \\
	a_{21} & a_{22} & \cdots & a_{2n} & 0 & 1 & \cdots & 0 \\
	\vdots & \vdots & \ddots & \vdots & \vdots & \vdots & \ddots & \vdots \\
	a_{m1} & a_{m2} & \cdots & a_{mn} & 0 & 0 & \cdots & 1
	\end{bmatrix} .
	\end{align*}
	
\noindent Now, the goal, if possible, is to perform row operations to change the left-side (the matrix $A$) into the identity matrix, that is:
	\begin{align*}
	\begin{bmatrix}
	I & B
	\end{bmatrix}
	=
	\begin{bmatrix}
	1 & 0 & \cdots & 0 & b_{11} & b_{12} & \cdots & b_{1n} \\
	0 & 1 & \cdots & 0 & b_{21} & b_{22} & \cdots & b_{2n}  \\
	\vdots & \vdots & \ddots & \vdots & \vdots & \vdots & \ddots & \vdots \\
	0 & 0 & \cdots & 1 & b_{m1} & b_{m2} & \cdots & b_{mn}
	\end{bmatrix} .
	\end{align*}

\noindent\underline{Remark}: 
	\begin{itemize}
	\item When it's possible to transform the augmented matrix $[A \quad I ]$ into the augmented matrix $[I \quad B ]$, then $B$ is the inverse of $A$.
	\item When it's not possible to transform $[ A \quad I]$ into $[I \quad B]$, then $A$ is singular. 
	\end{itemize}

\vspace*{16pt}

\begin{example}\label{Example:InvertableMatrix}
If possible, find the inverse of the following matrix:
	\begin{align*}
	A = \begin{bmatrix}
	2 & 1 & 3 \\ 2 & 1 & 1 \\ 4 & 5 & 1
	\end{bmatrix} .
	\end{align*}
\end{example}

\newpage

\phantom{2}

\newpage

\begin{example}
If possible, find the inverse of the following matrix:
	\begin{align*}
	A = \begin{bmatrix}
	1 & -2 & 2 \\
	2 & -3 & 1 \\
	1 & -1 & -1
	\end{bmatrix} .
	\end{align*}
\end{example}

\newpage

\subsection{Inverses to Solve Systems}
If you have a given system of linear equations
	\begin{align*}
	AX = B
	\end{align*}
where $A$ is a nonsingular matrix, then you can find $X$ (the vector of solutions) by multiplying on the left the whole equation by the inverse $A^{-1}$:
	\begin{align*}
	A^{-1} A X = A^{-1} B  \quad \Rightarrow \quad X = A^{-1} B .
	\end{align*}

\begin{example}
Solve the system
	\begin{align*}
	2x + y + 3z = 6 \\
	2x + y + z = -12 \\
	4x + 5y + z = 3 .
	\end{align*}
\end{example}

\newpage


\section{Elementary Matrices}
When we are performing row operations, we are in fact performing matrix multiplication with special matrices that we call elementary matrices.

\subsection{Three types}

	\begin{itemize}
	\item An elementary matrix obtained by interchanging two rows of $I$.
	\item An elementary matrix obtained by multiplying a row $I$ by a nonzero number.
	\item An elementary matrix obtained by replacing a row of $I$ by itself plus a multiple of another row of $I$.
	\end{itemize}

	\begin{example}
	Here are some examples of dimensions $3 \times 3$:
		\begin{align*}
		E_1 = \begin{bmatrix}
		0 & 1 & 0\\ 1 & 0 & 0 \\ 0 & 0 & 1
		\end{bmatrix}
		\quad
		E_2 = \begin{bmatrix}
		2 & 0 & 0 \\ 0 & 0 & 1 \\ 0 & 1 & 0
		\end{bmatrix}
		\quad
		E_3 = \begin{bmatrix}
		1 & 2 & 0 \\ 0 & 1 & 0 \\ 0 & 0 & 1
		\end{bmatrix} .
		\end{align*}
	\end{example}
	
\newpage

\subsection{Some mysteries Unraveled!}
When we were performing row operations on a matrix $A$, we were in fact performing a multiplication of an elementary matrix with $A$. Here are some facts related to this:
	\begin{itemize}
		\item If $E$ is obtained by interchanging rows $i$ and $j$ of $I$, then $EA$ is the matrix obtained from $A$ by interchanging rows $i$ and $j$ of $A$.
		\item If $E$ is obtained by multiplying row $i$ of $I$ by a scalar $c$, then $EA$ is the matrix obtained from $A$ by multiplying row $i$ of $A$ by $c$.
		\item If $E$ is obtained by replacing row $i$ of $I$ by itself plus $c$ times the row $j$ of $I$, then $EA$ is the matrix obtained from $A$ by replacing row $i$ of $A$ by itself plus $c$ times row $j$ of $A$.
	\end{itemize}
	
	\vspace*{12pt}
	
	\begin{example}
	Give the elementary matrices used in Example \ref{Example:InvertableMatrix}. At each step, using the elementary matrices, give the expression of the matrix resulting from the row operations.
	\end{example}
	
\newpage

\phantom{2}

\vfill

\newpage
	
\subsection{Inverses of elementary matrices}

\begin{example}
Consider the following elementary matrices
	\begin{align*}
	\begin{bmatrix}
	0 & 1 & 0 \\
	1 & 0 & 0 \\
	0 & 0 & 1
	\end{bmatrix} , \quad 
	\begin{bmatrix}
	3 & 0 & 0 \\
	0 & 1 & 0 \\
	0 & 0 & 1
	\end{bmatrix} , \quad
	\begin{bmatrix}
	3 & 1 & 0 \\
	0 & 1 & 0 \\
	0 & 0 & 1 
	\end{bmatrix} .
	\end{align*} 
For each of them, find the inverse.
\end{example}

\vfill

\noindent\underline{Remarks}: In general, if $E$ is an elementary matrix, then $E$ is invertible and:
	\begin{itemize}
	\item If $E$ is obtained by interchanging two rows of $I$, then $E^{-1} = E$;
	\item If $E$ is obtained by multiplying row $i$ of $I$ by a nonzero scalar $c$, then $E^{-1}$ is the matrix obtained by multiplying row $i$ of $I$ by $1/c$;
	\item If $E$ is obtained by replacing row $i$ of $I$ by itself plus $c$ times row $j$ of $I$, then $E^{-1}$ is the matrix obtained by replacing row $i$ of $I$ by itself plus $-c$ times row $j$ of $I$.
	\end{itemize}
	
\noindent\underline{Consequences}:
	\begin{itemize}
	\item A square matrix $A$ is invertible if and only if $A$ is a product of elementary matrices.
	\end{itemize}

\end{document}