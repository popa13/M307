\documentclass[12pt,a4paper]{article}
\usepackage[utf8]{inputenc}
\usepackage[english]{babel}

\usepackage{amsmath}
\usepackage{amsfonts}
\usepackage{amssymb}

\usepackage{graphicx}
\usepackage{lmodern}
\usepackage{tikz}
\usepackage{titlesec}
\usepackage{environ}
\usepackage{xcolor}
\usepackage{fancyhdr}
\usepackage[colorlinks = true, linkcolor = black]{hyperref}
\usepackage{xparse}
\usepackage{enumerate}

\usepackage[left=2cm,right=2cm,top=2cm,bottom=2cm]{geometry}
\usepackage{multicol}

\newcommand{\spaceP}{\vspace*{0.5cm}}

%% Redefining sections
\newcommand{\sectionformat}[1]{%
    \begin{tikzpicture}[baseline=(title.base)]
        \node[rectangle, draw] (title) {#1};
    \end{tikzpicture}
    
    \noindent\hrulefill
}

% default values copied from titlesec documentation page 23
% parameters of \titleformat command are explained on page 4
\titleformat%
    {\section}% <command> is the sectioning command to be redefined, i. e., \part, \chapter, \section, \subsection, \subsubsection, \paragraph or \subparagraph.
    {\normalfont\large\scshape}% <format>
    {}% <label> the number
    {0em}% <sep> length. horizontal separation between label and title body
    {\centering\sectionformat}% code preceding the title body  (title body is taken as argument)

%% Set counters for sections to none
\setcounter{secnumdepth}{0}

%% Set the footer/headers
\pagestyle{fancy}
\fancyhf{}
\renewcommand{\headrulewidth}{0pt}
\renewcommand{\footrulewidth}{2pt}
\lfoot{P.-O. Paris{\'e}}
\cfoot{MATH 307}
\rfoot{Page \thepage}

%% Defining example environment
\newcounter{example}[section]
\NewEnviron{example}%
	{%
	\noindent\refstepcounter{example}\fcolorbox{gray!40}{gray!40}{\textsc{\textcolor{red}{Example~\theexample.}}}%
	%\fcolorbox{black}{white}%
		{  %\parbox{0.95\textwidth}%
			{
			\BODY
			}%
		}%
	}

% Theorem environment
\NewEnviron{theorem}%
	{%
	\noindent\refstepcounter{example}\fcolorbox{gray!40}{gray!40}{\textsc{\textcolor{blue}{Theorem~\theexample.}}}%
	%\fcolorbox{black}{white}%
		{  %\parbox{0.95\textwidth}%
			{
			\BODY
			}%
		}%
	}

%% Commands for matrix of dimensions m x n
%\newcommand{\matMN}[1]{%
%	\begin{bmatrix}
%	#1_{11} & #1_{12} & #1_{13} & \cdots & #1_{1n} \\
%	#1_{21} & #1_{22} & #1_{23} & \cdots & #1_{2n} \\
%	#1_{31} & #1_{32} & #1_{33} & \cdots & #1_{3n} \\
%	\vdots & \vdots & \vdots & \ddots & \vdots \\
%	#1_{m1} & #1_{m2} & #1_{m3} & \cdots & #1_{mn}
%	\end{bmatrix}		
%	}
	
\NewDocumentCommand{\matMN}{mg}{%
	\IfNoValueTF{#2}
		{%
		  \begin{bmatrix}
		   #1_{11} & #1_{12} & #1_{13} & \cdots & #1_{1n} \\
	       #1_{21} & #1_{22} & #1_{23} & \cdots & #1_{2n} \\
		   #1_{31} & #1_{32} & #1_{33} & \cdots & #1_{3n} \\
		   \vdots & \vdots & \vdots & \ddots & \vdots \\
		   #1_{m1} & #1_{m2} & #1_{m3} & \cdots & #1_{mn}
		   \end{bmatrix}	%
		}
		{%
		   \begin{bmatrix}
		   #1_{11} + #2_{11} & #1_{12} + #2_{12} & \cdots & #1_{1n} + #2_{2n} \\
		   #1_{21} + #2_{21} & #1_{22} + #2_{22} & \cdots & #1_{2n} + #2_{2n} \\
		   \vdots & \vdots & \vdots & \ddots & \vdots \\
		   #1_{m1} + #2_{m1} & #1_{m2} + #2_{m2} & \cdots & #1_{mn} + #2_{mn}
		   \end{bmatrix}
		}%
}

%%%%
\begin{document}
\thispagestyle{empty}

\begin{center}
\vspace*{2.5cm}

{\Huge \textsc{Math 307}}

\vspace*{2cm}

{\LARGE \textsc{Chapter 1}} 

\vspace*{0.75cm}

\noindent\textsc{Section 1.6: Further Properties of Determinants}

\vspace*{0.75cm}

\tableofcontents

\vfill

\noindent \textsc{Created by: Pierre-Olivier Paris{\'e}} \\
\textsc{Summer 2022}
\end{center}

\newpage

\section{Determinants and Algebraic Operations}

	\subsection{Invertibility}
	
	A square matrix $A$ is invertible if and only if $\det (A) \neq 0$.
	
	\vspace*{16pt}
	
	\noindent\underline{Remark}: 
	\begin{itemize}
	\item If $A$ is an invertible matrix, then $\det (A^{-1}) = 1/\det (A)$.
	\item Determinant can help to determine if a system of linear equations has a solution or not.
	\end{itemize}
	
	\vspace*{16pt}
	
	\begin{example}
	Which of the following matrices are invertible:
		\begin{align*}
		A = \begin{bmatrix}
		1 & 2 \\
		2 & 3 
		\end{bmatrix}, 
		\quad
		B = \begin{bmatrix}
		1 & 2 & 3 & 4 \\
		-4 & -3 & -2 & 1 \\
		0 & 0 & 0 & 0 \\
		1 & 4 & 5 & 9
		\end{bmatrix}
		\quad \text{ and } \quad
		C = \begin{bmatrix}
		1 & 2 & -1 \\
		0 & 3 & -1 \\
		0 & 0 & 4
		\end{bmatrix} .
		\end{align*}
	\end{example}
	
	\newpage
	
	\subsection{Matrix Multiplication}
	
	If $A$ and $B$ are two $n \times n$ matrices, then
		\begin{align*}
		\det (AB ) = \det (A) \det (B) .
		\end{align*}
	
	\begin{example}
	Knowing that $\det (A) = 2$ and $\det (AB ) = 32$, find the determinant of the matrix $B$.
	\end{example}
	
	\vspace*{10cm}
	
	\subsection{Transpose}
	If $A$ is a square matrix, then $\det (A^\top ) = \det (A)$.
	
	\vspace*{16pt}
	
	\begin{example}
	If $A$ is the matrix
		\begin{align*}
		A = \begin{bmatrix}
		1 & 2 & 3 \\
		0 & 5 & -2 \\
		0 & 0 & 1
		\end{bmatrix}, 
		\end{align*}
	then find the determinant of $AA^{\top}$.
	\end{example}
	
\newpage

\section{Ajoint of a Matrix}

	\subsection{Matrix of Cofactors}
	
	The cofactor matrix is the matrix $C$ of all the cofactors of a given matrix $A$. If $A$ has dimensions $n \times n$, then
		\begin{align*}
		C = \begin{bmatrix}
		C_{11} & C_{12} & \cdots & C_{1n} \\
		C_{21} & C_{22} & \cdots & C_{2n} \\
		\vdots & \vdots & \ddots & \vdots \\
		C_{n1} & C_{n2} & \cdots & C_{nn} 
		\end{bmatrix} 
		\end{align*}
	where each $C_{ij}$ is the cofactor obtained from $A$.
	
	\vspace*{16pt}
	
	\begin{example}\label{Example:CofactorMatrix}
	Find the matrix of cofactors of the following matrix
		\begin{align*}
		A = \begin{bmatrix}
		-2 & 3 & 0 \\
		4 & 10 & 2 \\
		-5 & 7 & 0
		\end{bmatrix} .
		\end{align*}
	\end{example}
	
	\newpage
	
	\subsection{Definition of the Adjoint}
	The adjoint of a matrix $A$ of dimensions $n \times n$ is the transpose of the cofactor matrix. 
	
	\vspace*{16pt}
	
	\noindent Explicitly, we denote the adjoint of $A$ by $\mathrm{adj}\, (A)$ and its expression is
	\begin{align*}
	\mathrm{adj} (A) = C^{\top} = \begin{bmatrix}
		C_{11} & C_{21} & \cdots & C_{n1} \\
		C_{12} & C_{22} & \cdots & C_{n2} \\
		\vdots & \vdots & \ddots & \vdots \\
		C_{1n} & C_{2n} & \cdots & C_{nn} 
		\end{bmatrix}  .
	\end{align*}
	
	\begin{example}
	Find the adjoint of the matrix in Example \ref{Example:CofactorMatrix}.
	\end{example}
	
	\vfill
	
	\subsection{Another Way to Find the Inverse}
	If $A$ is an invertible matrix, then	
		\begin{align*}
		A^{-1} = \frac{1}{\det (A)} \mathrm{adj} \, (A) .
		\end{align*}
	
	\begin{example}
	Find the inverse of the matrix in Example \ref{Example:CofactorMatrix}.
	\end{example}
	
	\vfill
	
\newpage

\section{Cramer's Rule}
	
	Suppose that $AX = B$ is a system of $n$ linear equations in $n$ unknowns such that $\det (A) \neq 0$. Let 
		\begin{itemize}
		\item $A_1$ be the matrix obtained from $A$ by replacing the first column of $A$ by $B$;
		\item $A_2$ be the matrix obtained from $A$ by replacing the second column of $A$ by $B$.
		\item etc.
		\end{itemize}
	Then the solutions to the system are
		\begin{align*}
		x_1 = \frac{\det (A_1)}{\det (A)} , \quad x_2 = \frac{\det (A_2)}{\det (A)} , \quad \ldots , \quad x_n = \frac{\det (A_n)}{\det (A)} .
		\end{align*}

	
	\begin{example}
	Use Cramer's rule to solve the system
		\begin{align*}
		-2x + 3y \phantom{+ 2z} = 2 \\
		4x + 10y + 2z = 3 \\
		-5x + 7y \phantom{+ 2z} = 1 .
		\end{align*}
	\end{example}

\end{document}