\documentclass[12pt,a4paper]{article}
\usepackage[utf8]{inputenc}
\usepackage[english]{babel}

\usepackage{amsmath}
\usepackage{amsfonts}
\usepackage{amssymb}

\usepackage{graphicx}
\usepackage{lmodern}
\usepackage{tikz}
\usepackage{titlesec}
\usepackage{environ}
\usepackage{xcolor}
\usepackage{fancyhdr}
\usepackage[colorlinks = true, linkcolor = black]{hyperref}
\usepackage{xparse}
\usepackage{enumerate}

\usepackage[left=2cm,right=2cm,top=2cm,bottom=2cm]{geometry}
\usepackage{multicol}
\usepackage[indent=0pt]{parskip}

\newcommand{\spaceP}{\vspace*{0.5cm}}
\newcommand{\Span}{\mathrm{Span}\,}
\newcommand{\range}{\mathrm{range}\,}

%% Redefining sections
\newcommand{\sectionformat}[1]{%
    \begin{tikzpicture}[baseline=(title.base)]
        \node[rectangle, draw] (title) {#1};
    \end{tikzpicture}
    
    \noindent\hrulefill
}

% default values copied from titlesec documentation page 23
% parameters of \titleformat command are explained on page 4
\titleformat%
    {\section}% <command> is the sectioning command to be redefined, i. e., \part, \chapter, \section, \subsection, \subsubsection, \paragraph or \subparagraph.
    {\normalfont\large\scshape}% <format>
    {}% <label> the number
    {0em}% <sep> length. horizontal separation between label and title body
    {\centering\sectionformat}% code preceding the title body  (title body is taken as argument)

%% Set counters for sections to none
\setcounter{secnumdepth}{0}

%% Set the footer/headers
\pagestyle{fancy}
\fancyhf{}
\renewcommand{\headrulewidth}{0pt}
\renewcommand{\footrulewidth}{2pt}
\lfoot{P.-O. Paris{\'e}}
\cfoot{MATH 307}
\rfoot{Page \thepage}

%% Defining example environment
\newcounter{example}[section]
\NewEnviron{example}%
	{%
	\noindent\refstepcounter{example}\fcolorbox{gray!40}{gray!40}{\textsc{\textcolor{red}{Example~\theexample.}}}%
	%\fcolorbox{black}{white}%
		{  %\parbox{0.95\textwidth}%
			{
			\BODY
			}%
		}%
	}

% Theorem environment
\NewEnviron{theorem}%
	{%
	\noindent\refstepcounter{example}\fcolorbox{gray!40}{gray!40}{\textsc{\textcolor{blue}{Theorem~\theexample.}}}%
	%\fcolorbox{black}{white}%
		{  %\parbox{0.95\textwidth}%
			{
			\BODY
			}%
		}%
	}

%% Commands for matrix of dimensions m x n
%\newcommand{\matMN}[1]{%
%	\begin{bmatrix}
%	#1_{11} & #1_{12} & #1_{13} & \cdots & #1_{1n} \\
%	#1_{21} & #1_{22} & #1_{23} & \cdots & #1_{2n} \\
%	#1_{31} & #1_{32} & #1_{33} & \cdots & #1_{3n} \\
%	\vdots & \vdots & \vdots & \ddots & \vdots \\
%	#1_{m1} & #1_{m2} & #1_{m3} & \cdots & #1_{mn}
%	\end{bmatrix}		
%	}
	
\NewDocumentCommand{\matMN}{mg}{%
	\IfNoValueTF{#2}
		{%
		  \begin{bmatrix}
		   #1_{11} & #1_{12} & #1_{13} & \cdots & #1_{1n} \\
	       #1_{21} & #1_{22} & #1_{23} & \cdots & #1_{2n} \\
		   #1_{31} & #1_{32} & #1_{33} & \cdots & #1_{3n} \\
		   \vdots & \vdots & \vdots & \ddots & \vdots \\
		   #1_{m1} & #1_{m2} & #1_{m3} & \cdots & #1_{mn}
		   \end{bmatrix}	%
		}
		{%
		   \begin{bmatrix}
		   #1_{11} + #2_{11} & #1_{12} + #2_{12} & \cdots & #1_{1n} + #2_{2n} \\
		   #1_{21} + #2_{21} & #1_{22} + #2_{22} & \cdots & #1_{2n} + #2_{2n} \\
		   \vdots & \vdots & \vdots & \ddots & \vdots \\
		   #1_{m1} + #2_{m1} & #1_{m2} + #2_{m2} & \cdots & #1_{mn} + #2_{mn}
		   \end{bmatrix}
		}%
}

%%%%
\begin{document}
\thispagestyle{empty}

\begin{center}
\vspace*{2.5cm}

{\Huge \textsc{Math 307}}

\vspace*{2cm}

{\LARGE \textsc{Chapter 6}} 

\vspace*{0.75cm}

\noindent\textsc{Section 6.2: Homogeneous Systems With Constant Coefficients \\The Diagonalizable Case}

\vspace*{0.75cm}

\tableofcontents

\vfill

\noindent \textsc{Created by: Pierre-Olivier Paris{\'e}} \\
\textsc{Summer 2022}
\end{center}

\newpage

\section{Real Eigenvalues}

	\begin{example}
	Determine the general solution to
		\begin{align*}
		Y' = \begin{bmatrix}
		1 & -3 \\ -2 & 2
		\end{bmatrix} Y. 
		\end{align*}
	\end{example}
	
	\vfill
	
	\underline{Fact}: Suppose $A$ and $B$ are $n \times n$ matrices with $B = P^{-1} A P$ for some invertible $n \times n$ matrix $P$. Then
		\begin{itemize}
		\item If $Z$ is a solution to $Y' = BY$, then $PZ$ is a solution to $Y' = AY$.
		\item If $Z_1$, $Z_2$, $\ldots$, $Z_n$ is a fundamental set of solutions of $Y' = BY$, then $PZ_1$, $PZ_2$, $\ldots$, $PZ_n$ is a fundamental set of solutions to $Y' = AY$.
		\end{itemize}
		
	\newpage
	
	\begin{example}
	Solve the initial value problem
		\begin{align*}
		Y' = \begin{bmatrix}
		2 & -3 & -3 \\
		2 & -2 & -2 \\
		-2 & 1 & 1
		\end{bmatrix}Y , \quad Y(0) = \begin{bmatrix}
		1 \\ 0 \\ 0
		\end{bmatrix} .
		\end{align*}
	\end{example}
	
	\newpage
	
\section{Imaginary Eigenvalues}

	\subsection{Complex Exponential Function}
	For a complex number $z = a + ib$, we define
		\begin{align*}
		e^z = e^{a + ib} = e^a \cos (b) + i e^a \sin (b) .
		\end{align*}
		
	The solution to the differential equation $y' = (a + ib) y$ is
		\begin{align*}
		y (x) = e^{(a + ib)x} .
		\end{align*}
	
	\subsection{Finding solutions with complex numbers}
	
	\begin{example}
	Find the general solution to 
		\begin{align*}
		Y' = \begin{bmatrix}
		1 & -1 \\ 1 & 1
		\end{bmatrix} Y.
		\end{align*}
	\end{example}
	
	\newpage
	
	\phantom{p}
	
	\vfill
	
	\underline{Fact}: If $U(x) + i V(x)$ is a solution to $Y' = A Y$, then $U(x)$ and $V(x)$ are solutions to $Y' = A Y$.
	
	\newpage
	
	\begin{example}
	Find the general solution to
		\begin{align*}
		Y' = \begin{bmatrix}
		1 & 1 & 1 \\
		0 & 1 & -1 \\
		0 & 1 & 1
		\end{bmatrix} Y.
		\end{align*}
	\end{example}

\end{document}