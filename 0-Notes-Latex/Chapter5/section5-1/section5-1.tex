\documentclass[12pt,a4paper]{article}
\usepackage[utf8]{inputenc}
\usepackage[english]{babel}

\usepackage{amsmath}
\usepackage{amsfonts}
\usepackage{amssymb}

\usepackage{graphicx}
\usepackage{lmodern}
\usepackage{tikz}
\usepackage{titlesec}
\usepackage{environ}
\usepackage{xcolor}
\usepackage{fancyhdr}
\usepackage[colorlinks = true, linkcolor = black]{hyperref}
\usepackage{xparse}
\usepackage{enumerate}

\usepackage[left=2cm,right=2cm,top=2cm,bottom=2cm]{geometry}
\usepackage{multicol}
\usepackage[indent=0pt]{parskip}

\newcommand{\spaceP}{\vspace*{0.5cm}}
\newcommand{\Span}{\mathrm{Span}\,}
\newcommand{\range}{\mathrm{range}\,}

%% Redefining sections
\newcommand{\sectionformat}[1]{%
    \begin{tikzpicture}[baseline=(title.base)]
        \node[rectangle, draw] (title) {#1};
    \end{tikzpicture}
    
    \noindent\hrulefill
}

% default values copied from titlesec documentation page 23
% parameters of \titleformat command are explained on page 4
\titleformat%
    {\section}% <command> is the sectioning command to be redefined, i. e., \part, \chapter, \section, \subsection, \subsubsection, \paragraph or \subparagraph.
    {\normalfont\large\scshape}% <format>
    {}% <label> the number
    {0em}% <sep> length. horizontal separation between label and title body
    {\centering\sectionformat}% code preceding the title body  (title body is taken as argument)

%% Set counters for sections to none
\setcounter{secnumdepth}{0}

%% Set the footer/headers
\pagestyle{fancy}
\fancyhf{}
\renewcommand{\headrulewidth}{0pt}
\renewcommand{\footrulewidth}{2pt}
\lfoot{P.-O. Paris{\'e}}
\cfoot{MATH 307}
\rfoot{Page \thepage}

%% Defining example environment
\newcounter{example}[section]
\NewEnviron{example}%
	{%
	\noindent\refstepcounter{example}\fcolorbox{gray!40}{gray!40}{\textsc{\textcolor{red}{Example~\theexample.}}}%
	%\fcolorbox{black}{white}%
		{  %\parbox{0.95\textwidth}%
			{
			\BODY
			}%
		}%
	}

% Theorem environment
\NewEnviron{theorem}%
	{%
	\noindent\refstepcounter{example}\fcolorbox{gray!40}{gray!40}{\textsc{\textcolor{blue}{Theorem~\theexample.}}}%
	%\fcolorbox{black}{white}%
		{  %\parbox{0.95\textwidth}%
			{
			\BODY
			}%
		}%
	}

%% Commands for matrix of dimensions m x n
%\newcommand{\matMN}[1]{%
%	\begin{bmatrix}
%	#1_{11} & #1_{12} & #1_{13} & \cdots & #1_{1n} \\
%	#1_{21} & #1_{22} & #1_{23} & \cdots & #1_{2n} \\
%	#1_{31} & #1_{32} & #1_{33} & \cdots & #1_{3n} \\
%	\vdots & \vdots & \vdots & \ddots & \vdots \\
%	#1_{m1} & #1_{m2} & #1_{m3} & \cdots & #1_{mn}
%	\end{bmatrix}		
%	}
	
\NewDocumentCommand{\matMN}{mg}{%
	\IfNoValueTF{#2}
		{%
		  \begin{bmatrix}
		   #1_{11} & #1_{12} & #1_{13} & \cdots & #1_{1n} \\
	       #1_{21} & #1_{22} & #1_{23} & \cdots & #1_{2n} \\
		   #1_{31} & #1_{32} & #1_{33} & \cdots & #1_{3n} \\
		   \vdots & \vdots & \vdots & \ddots & \vdots \\
		   #1_{m1} & #1_{m2} & #1_{m3} & \cdots & #1_{mn}
		   \end{bmatrix}	%
		}
		{%
		   \begin{bmatrix}
		   #1_{11} + #2_{11} & #1_{12} + #2_{12} & \cdots & #1_{1n} + #2_{2n} \\
		   #1_{21} + #2_{21} & #1_{22} + #2_{22} & \cdots & #1_{2n} + #2_{2n} \\
		   \vdots & \vdots & \vdots & \ddots & \vdots \\
		   #1_{m1} + #2_{m1} & #1_{m2} + #2_{m2} & \cdots & #1_{mn} + #2_{mn}
		   \end{bmatrix}
		}%
}

%%%%
\begin{document}
\thispagestyle{empty}

\begin{center}
\vspace*{2.5cm}

{\Huge \textsc{Math 307}}

\vspace*{2cm}

{\LARGE \textsc{Chapter 5}} 

\vspace*{0.75cm}

\noindent\textsc{Section 5.1: Linear Transformations}

\vspace*{0.75cm}

\tableofcontents

\vfill

\noindent \textsc{Created by: Pierre-Olivier Paris{\'e}} \\
\textsc{Summer 2022}
\end{center}

\newpage

\section{What is a Linear Transformation?}
\underline{Convention}:

The addition and scalar multiplication on the set of column vectors $\mathbb{R}^n$ are the usual ones that make $\mathbb{R}^n$ a vector space. If the addition is changed, it will be mentioned explicitly in the text.

	\subsection{Definition}
	If $V$ and $W$ are vector spaces, a function $T : V \rightarrow W$ is called a \textbf{linear transformation} if, for all vectors $u$ and $v$ in $V$ and all scalars $c$, the following two properties are satisfied:
		\begin{enumerate}
		\item $T (u + v) = T(u) + T(v)$;
		\item $T(cv) = c T(v)$.
		\end{enumerate}
		
	\vspace*{12pt}

	\begin{example}
	Let $A$ be an $m \times n$ matrix. We define $T : \mathbb{R}^n \rightarrow \mathbb{R}^m$ by
		\begin{align*}
		T (X) := A X
		\end{align*}
	where $X$ is an $n \times 1$ column vector. Verify that the function $T$ is a linear transformation.
	\end{example}
	
	\newpage
	
	\begin{example}
	Verify if the given function $T : \mathbb{R}^3 \rightarrow \mathbb{R}^2$ defined by
		\begin{align*}
		T \left( \begin{bmatrix}
		x \\ y \\ z
		\end{bmatrix} \right) = \begin{bmatrix}
		x + y - z \\ x + 2y + z
		\end{bmatrix}
		\end{align*}
	is a linear transformation.
	\end{example}
	
	\newpage
	
	\begin{example}
	Let $D(a, b)$ be the subspace of $F(a, b)$ of differentiable function on the interval $(a, b)$. Define the function $T : D (a, b) \rightarrow F (a, b)$ by
		\begin{align*}
		T (f) := f'
		\end{align*}
	meaning that $T (f) (x) = f'(x)$ for every $x$ in $(a, b)$. Verify that $T$ is a linear transformation.
	\end{example}
	
	\vfill
	
	\underline{Remark}: The linear transformation in the previous example is called a differential operator and is quite useful in the theory of ODE and PDE.
	
	\newpage
	
	\section{Basic Properties}
	If $T : V \rightarrow W$ is a linear transformation, then we can prove that
		\begin{itemize}
		\item $T(0) = 0$;
		\item $T(-v) = -T(v)$ for any vector $v$ in $V$;
		\item $T(u - v) = T(u) - T(v)$ for any vector $u$, $v$ in $V$.
		\end{itemize}
		
	There is another important property of a linear transformation which we shall illustrate by an example.
	
	\begin{example}
	Suppose that $T : \mathbb{R}^3 \rightarrow \mathbb{R}^2$ is a linear transformation so that
		\begin{align*}
		T \left( \begin{bmatrix}
		1 \\ 1 \\ 0
		\end{bmatrix} \right) = \begin{bmatrix}
		2 \\ 3
		\end{bmatrix}
		\text{, } \quad
		T \left( \begin{bmatrix}
		0 \\ 1 \\ 1
		\end{bmatrix} \right) = \begin{bmatrix}
		0 \\ 3
		\end{bmatrix}
		\quad \text{ and } \quad
		T \left( \begin{bmatrix}
		1 \\ 0 \\ 1
		\end{bmatrix} \right) = \begin{bmatrix}
		0 \\ 2
		\end{bmatrix} .
		\end{align*}
	Find the value of $T \left( \begin{bmatrix}
		1 \\ 3 \\ 0
		\end{bmatrix} \right) $. 
	\end{example}
	
	\vfill
	
	\underline{Fact}: If $v_1$, $v_2$, $\ldots$, $v_n$ form a basis, then the values of a linear transformation $T$ is determined by its value on $v_1$, $v_2$, $\ldots$, $v_n$ because for any $v \in V$, we have
		\begin{align*}
		T(v) = T(c_1 v_1 + c_2 v_2 + \cdots + c_n v_n) = c_1 T(v_1 ) + c_2 T(v_2) + \cdots + c_n T(v_n) .
		\end{align*}
		
	\newpage
		
	\section{Subspaces of a Linear Transformation}
	
	\subsection{Kernel}
	If $T : V \rightarrow W$ is a linear transformation, then the \textbf{kernel} of $T$ is the set of all vectors $v$ in $V$ such that $T(v) = 0$. In set notation:
		\begin{align*}
		\ker (T) = \{ v \in V \, : \, T(v) = 0 \} .
		\end{align*}
		
	This is in general a subspace of $V$.
		
	\begin{example}\label{Ex:kernelBasis}
	Find a basis for the kernel of the linear transformation
		\begin{align*}
		T \left( \begin{bmatrix}
		x \\ y \\ z
		\end{bmatrix} \right) = \begin{bmatrix}
		x + y - z \\ x + 2y + z
		\end{bmatrix} .
		\end{align*}
	\end{example}
	
	\vfill
	
	\underline{Remark}: The kernel of a transformation is related to the solutions of the system of linear equations $AX = 0$ when $T (X) = AX$ with $A$ an $m \times n$ matrix. In this particular situation, the kernel $\ker (T)$ is called the \textbf{null space} of $A$ also denoted by $NS (A)$. In other words, we have
		\begin{align*}
		NS (A) = \ker (T) .
		\end{align*}
		
	\newpage
	
	\subsection{Range}
	If $T : V \rightarrow W$ is a linear transformation, then the \textbf{range} of $T$ is the set of all vectors $T(v)$ where $v$ is in $V$. In set notation:
		\begin{align*}
		\range (T) = \{  T(v) \, : \, v \in V \} .
		\end{align*}
	
	This is in general a subspace of $W$.
	
	\underline{Facts}: 
		\begin{itemize}
		\item Finding a basis for the range of a tranformation $T$ given by $T(X) = AX$ where $A$ is an $m \times n$ matrix is equivalent to finding a basis for the spanning set of the columns of the matrix $A$.
		\item The subspace spans by the column of a matrix $A$ is called the \textbf{column space} and is denoted by $CS (A)$.
		\end{itemize}
		
	\begin{example}\label{Ex:RangeBasis}
	Find a basis for the range of the linear transformation of Example \ref{Ex:kernelBasis} using the column space of a certain matrix.
	\end{example}
	
	\newpage
	
	In summary, to find $\range (T)$ or $CS (A)$ for a linear transformation of the form $T(X) = AX$, we follow these steps:
		\begin{itemize}
		\item express $T(v)$ as a linear combination of column vectors $v_1$, $v_2$, $\ldots$, $v_n$.
		\item Write each vector in a matrix $A = \begin{bmatrix} v_1 & v_2 & \cdots & v_n \end{bmatrix}$.
		\item Find the RREF of $A$.
		\item The column with the first $1$ in a row will be a pivot and the vector corresponding to the column will be part of the basis.
		\end{itemize}
	
	\underline{Fact}: We call $\dim (CS (A)) = \dim (\range (T))$ the \textbf{rank} of the matrix $A$ or transformation $T$.
	
	\subsection{Rank-Nullity Identity}
	We define
		\begin{itemize}
		\item the \textbf{nullity} of a linear transformation $T$ as the dimension of $\ker (T)$.
		\item the \textbf{rank} of a linear transformation $T$ as the dimension of $\range (T)$.
		\end{itemize}
	 
	 Here is an important identity relating the rank and the nullity of a linear transformation.
	 
	 \vspace*{10pt}
	 
	 \begin{theorem}
	 If $T : V \rightarrow W$ is a linear transformation, then
	 	\begin{align*}
	 	\dim (\ker (T)) + \dim (\range (T)) = \dim (V) .
	 	\end{align*}
	 \end{theorem}
	 
	 \underline{Remark}: For an $m \times n$ matrix, we obtain
	 	\begin{align*}
	 	\dim (NS (A)) + \dim (CS (A)) = n .
	 	\end{align*}
	 
	 \vspace*{14pt}
	 
	 \begin{example}
	 Verify the Rank-Nullity Identity for the matrix in Example \ref{Ex:kernelBasis} and Example \ref{Ex:RangeBasis}.
	 \end{example}

\end{document}


\subsection{Row Subspace of a Matrix}
	Given an $m \times n$ matrix $A$, the subspace spanned by the rows of $A$ is called the \textbf{row space} of $A$ and is denoted by
		\begin{align*}
		RS (A) .
		\end{align*}
		
	\vspace*{12pt}
	
	We will say that a matrix $A$ is row equivalent to a matrix $B$ if we can obtain $B$ from $A$ by a sequence of row operations on $A$.
	
	\vspace*{12pt}
	
	\begin{theorem}
	If $A$ is row equivalent to $B$, then $RS (A) = RS (B)$.
	\end{theorem}
	
	\vspace*{12pt}
	
	\begin{example}
	Use the last Theorem to find a basis for the row space of the matrix in Example \ref{Ex:kernelBasis}.
	\end{example}
	
	\newpage