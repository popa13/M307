\documentclass[12pt]{article}
\usepackage[utf8]{inputenc}

\usepackage{lmodern}

\usepackage{enumitem}
\usepackage[margin=2cm]{geometry}

\usepackage{amsmath, amsfonts, amssymb}
\usepackage{graphicx}
\usepackage{tikz}
\usepackage{pgfplots}
\usepackage{multicol}

\usepackage{comment}
\usepackage{url}
\usepackage{calc}
\usepackage{subcaption}

\usepackage{array}
\usepackage{blkarray,booktabs, bigstrut}

\pgfplotsset{compat=1.16}

% MATH commands
\newcommand{\ga}{\left\langle}
\newcommand{\da}{\right\rangle}
\newcommand{\oa}{\left\lbrace}
\newcommand{\fa}{\right\rbrace}
\newcommand{\oc}{\left[}
\newcommand{\fc}{\right]}
\newcommand{\op}{\left(}
\newcommand{\fp}{\right)}

\newcommand{\bi}{\mathbf{i}}
\newcommand{\bj}{\mathbf{j}}
\newcommand{\bk}{\mathbf{k}}
\newcommand{\bF}{\mathbf{F}}

\newcommand{\ra}{\rightarrow}
\newcommand{\Ra}{\Rightarrow}

\newcommand{\sech}{\mathrm{sech}\,}
\newcommand{\csch}{\mathrm{csch}\,}
\newcommand{\curl}{\mathrm{curl}\,}
\newcommand{\dive}{\mathrm{div}\,}

\newcommand{\ve}{\varepsilon}
\newcommand{\spc}{\vspace*{0.5cm}}

\DeclareMathOperator{\Ran}{Ran}
\DeclareMathOperator{\Dom}{Dom}

\newcommand{\exo}[2]{\noindent\textcolor{red}{\fbox{\textbf{Section {#1}, Problem {#2}}}}\\}

\begin{document}
	\noindent \hrulefill \\
	MATH-307 \hfill Pierre-Olivier Paris{\'e}\\
	Homework 3 Solutions \hfill Summer 2022\\\vspace*{-0.7cm}
	
	\noindent\hrulefill
	
	\spc
	
	\exo{1.2}{2}
	\\
	We have
		\begin{align*}
		2B = 2 \left[\begin{matrix}2 & -1\\-3 & -2\\0 & 4\end{matrix}\right] = \left[\begin{matrix}4 & -2\\-6 & -4\\0 & 8\end{matrix}\right] .
		\end{align*}
		
	\newpage
	
	\exo{1.2}{9}
	\\
	We will first multiply $E$ by the first column of $F$:
		\begin{align*}
		E F_1 = \left[\begin{matrix}1 & -3 & 5\\2 & 1 & -1\\1 & 1 & 0\end{matrix}\right]\left[\begin{matrix}1\\2\\1\end{matrix}\right] = \left[\begin{matrix}0\\3\\3\end{matrix}\right] .
		\end{align*}
	Then we multiply $E$ by the second column of $F$:
		\begin{align*}
		E F_2 = \left[\begin{matrix}1 & -3 & 5\\2 & 1 & -1\\1 & 1 & 0\end{matrix}\right]\left[\begin{matrix}-1\\-3\\0\end{matrix}\right] = \left[\begin{matrix}8\\-5\\-4\end{matrix}\right] .
		\end{align*}
	Finally, we multiply $E$ by the third column of $F$:
		\begin{align*}
		E F_3 = \left[\begin{matrix}1 & -3 & 5\\2 & 1 & -1\\1 & 1 & 0\end{matrix}\right]\left[\begin{matrix}4\\6\\1\end{matrix}\right] = \left[\begin{matrix}-9\\13\\10\end{matrix}\right] .
		\end{align*}
	Putting each new columns together, we obtain
		\begin{align*}
		EF = \left[\begin{matrix}1 & -3 & 5\\2 & 1 & -1\\1 & 1 & 0\end{matrix}\right] \left[\begin{matrix}1 & -1 & 4\\2 & -3 & 6\\1 & 0 & 1\end{matrix}\right] = \left[\begin{matrix}0 & 8 & -9\\3 & -5 & 13\\3 & -4 & 10\end{matrix}\right] .
		\end{align*}
		
	\newpage
	
	\exo{1.2}{23}
	\begin{enumerate}
	\item Using the properties of the product over the addition:
		\begin{align*}
		(A + B)^2 = (A + B)(A + B) = AA + AB + BA + BB = A^2 + AB + BA + B^2 .
		\end{align*}
	\item Consider the following two matrices:
		\begin{align*}
		A = \left[\begin{matrix}1 & 0 & 0\\0 & 0 & 1\\0 & 1 & 0\end{matrix}\right] \quad \text{ and } \quad B = \left[\begin{matrix}1 & 2 & 3\\4 & 5 & 6\\7 & 8 & 9\end{matrix}\right] .
		\end{align*}
	Then we obtain
		\begin{align*}
		(A + B)^2 = \left(\left[\begin{matrix}1 & 0 & 0\\0 & 0 & 1\\0 & 1 & 0\end{matrix}\right] + \left[\begin{matrix}1 & 2 & 3\\4 & 5 & 6\\7 & 8 & 9\end{matrix}\right] \right)^2 = \left[\begin{matrix}33 & 41 & 47\\77 & 96 & 110\\113 & 140 & 165\end{matrix}\right] .
		\end{align*}
	However, we have
		\begin{align*}
		A^2 &= \left[\begin{matrix}1 & 0 & 0\\0 & 1 & 0\\0 & 0 & 1\end{matrix}\right] \\
		B^2 &= \left[\begin{matrix}30 & 36 & 42\\66 & 81 & 96\\102 & 126 & 150\end{matrix}\right] \\
		2 AB &= \left[\begin{matrix}2 & 4 & 6\\14 & 16 & 18\\8 & 10 & 12\end{matrix}\right] .
		\end{align*}
	So adding up together the previous results, we obtain
		\begin{align*}
		A^2 + B^2 + 2AB = \left[\begin{matrix}33 & 40 & 48\\80 & 98 & 114\\110 & 136 & 163\end{matrix}\right] .
		\end{align*}
	If we compare with $(A + B)^2$, then we see that
		\begin{align*}
		(A + B)^2 \neq A^2 + B^2 + 2AB .
		\end{align*}
	\end{enumerate}
	
	\newpage
	
	\exo{1.3}{3}
	\\
	To find the inverse of $A$, we augment $A$ with the $3 \times 3$ matrix. Then, we reduce the left-hand side to the identity.
		\begin{align*}
		\left[\begin{matrix}1 & -2 & 3 & 1 & 0 & 0\\2 & -1 & 4 & 0 & 1 & 0\\1 & 1 & 1 & 0 & 0 & 1\end{matrix}\right] & \sim \left[\begin{matrix}1 & -2 & 3 & 1 & 0 & 0\\0 & -3 & 2 & 2 & -1 & 0\\0 & 3 & -2 & -1 & 0 & 1\end{matrix}\right] \\
		& \sim \left[\begin{matrix}3 & 0 & 5 & -1 & 2 & 0\\0 & -3 & 2 & 2 & -1 & 0\\0 & 0 & 0 & 1 & -1 & 1\end{matrix}\right]
		\end{align*}
	We can see that there is a line starting with three zeros, but with $1$, $-1$, and $1$. Thus, the matrix $A$ has no inverse.
	
	\newpage
	
	\exo{1.3}{5}
	\\
	To find the inverse of $A$, we augment $A$ with the $3 \times 3$ matrix. Then, we reduce the left-hand side to the identity.
		\begin{align*}
		\left[\begin{matrix}0 & -2 & 1 & 1 & 0 & 0\\2 & 4 & -1 & 0 & 1 & 0\\2 & 1 & 2 & 0 & 0 & 1\end{matrix}\right] & \sim \left[\begin{matrix}2 & 1 & 2 & 0 & 0 & 1\\0 & 3 & -3 & 0 & 1 & -1\\0 & -2 & 1 & 1 & 0 & 0\end{matrix}\right] \\
		& \sim \left[\begin{matrix}6 & 0 & 9 & 0 & -1 & 4\\0 & 3 & -3 & 0 & 1 & -1\\0 & 0 & -3 & 3 & 2 & -2\end{matrix}\right] \\
		& \sim \left[\begin{matrix}6 & 0 & 0 & 9 & 5 & -2\\0 & 3 & 0 & -3 & -1 & 1\\0 & 0 & -3 & 3 & 2 & -2\end{matrix}\right] \\
		& \sim \left[\begin{matrix}1 & 0 & 0 & 3/2 & 5/6 & -1/3 \\0 & 1 & 0 & -1 & -1/3 & 1/3 \\0 & 0 & 1 & -1 & -2/3 & 2/3 \end{matrix}\right] .
		\end{align*}
	We can conclude that the matrix is invertible and its inverse is
		\begin{align*}
		A^{-1} = \left[\begin{matrix} 3/2 & 5/6 & -1/3 \\ -1 & -1/3 & 1/3 \\ -1 & -2/3 & 2/3 \end{matrix}\right] .
		\end{align*}
		
	\newpage
	
	\exo{1.3}{16}
	\\
	If $A$ has a row of zeros, this means the augmented matrix created out of $A$ and the identity matrix has the following form:
		\begin{align*}
		\begin{bmatrix}
		a_{11} & a_{12} & \cdots & a_{1n} & 1 & 0 &  \cdots & 0 & \cdots & 0 \\
		a_{21} & a_{22} & \cdots & a_{2n} & 0 & 1 & \cdots & 0 & \cdots & 0 \\
		\vdots & \vdots & \vdots & \vdots & \vdots & \vdots & \vdots & \vdots & \vdots & \vdots \\
		0 & 0 & \cdots & 0 & 0 & 0 & \cdots & 1 & \cdots & 0 \\
		\vdots & \vdots & \vdots & \vdots & \vdots & \vdots & \vdots & \vdots & \vdots & \vdots \\
		a_{n1} & a_{n2} & \cdots & a_{nn} & 0 & 0 & \cdots & 0 & \cdots & 1
		\end{bmatrix}
		\end{align*}
	Thus, there is a row with zeros which equals $1$. Thus the matrix can't have an inverse.
	
	If $A$ has a column of zeros, then $A^\top$ has a row of zeros. Since $A$ is invertible if and only if $A^\top$ is invertible, this implies that we can check if $A^\top$ is invertible. From the previous paragraph, since $A^\top$ has a row of zeros, then $A^\top$ is not invertible. So $A$ is not invertible.
	
\end{document}