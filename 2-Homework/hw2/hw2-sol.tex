\documentclass[12pt]{article}
\usepackage[utf8]{inputenc}

\usepackage{lmodern}

\usepackage{enumitem}
\usepackage[margin=2cm]{geometry}

\usepackage{amsmath, amsfonts, amssymb}
\usepackage{graphicx}
\usepackage{tikz}
\usepackage{pgfplots}
\usepackage{multicol}

\usepackage{comment}
\usepackage{url}
\usepackage{calc}
\usepackage{subcaption}

\usepackage{array}
\usepackage{blkarray,booktabs, bigstrut}

\pgfplotsset{compat=1.16}

% MATH commands
\newcommand{\ga}{\left\langle}
\newcommand{\da}{\right\rangle}
\newcommand{\oa}{\left\lbrace}
\newcommand{\fa}{\right\rbrace}
\newcommand{\oc}{\left[}
\newcommand{\fc}{\right]}
\newcommand{\op}{\left(}
\newcommand{\fp}{\right)}

\newcommand{\bi}{\mathbf{i}}
\newcommand{\bj}{\mathbf{j}}
\newcommand{\bk}{\mathbf{k}}
\newcommand{\bF}{\mathbf{F}}

\newcommand{\ra}{\rightarrow}
\newcommand{\Ra}{\Rightarrow}

\newcommand{\sech}{\mathrm{sech}\,}
\newcommand{\csch}{\mathrm{csch}\,}
\newcommand{\curl}{\mathrm{curl}\,}
\newcommand{\dive}{\mathrm{div}\,}

\newcommand{\ve}{\varepsilon}
\newcommand{\spc}{\vspace*{0.5cm}}

\DeclareMathOperator{\Ran}{Ran}
\DeclareMathOperator{\Dom}{Dom}

\newcommand{\exo}[2]{\noindent\textcolor{red}{\fbox{\textbf{Section {#1}, Problem {#2}}}}\\}

\begin{document}
	\noindent \hrulefill \\
	MATH-307 \hfill Pierre-Olivier Paris{\'e}\\
	Homework 2 Solutions \hfill Summer 2022\\\vspace*{-0.7cm}
	
	\noindent\hrulefill
	
	\spc
	
	\exo{1.1}{2}
	\\
	We have
		\begin{align*}
		\left[\begin{matrix}2 & 1 & -2 & 0\\2 & -1 & -2 & 0\\1 & 2 & -4 & 0\end{matrix}\right]
		& \sim 
		\left[\begin{matrix}2 & 1 & -2 & 0\\0 & 2 & 0 & 0\\0 & -3 & 6 & 0\end{matrix}\right] \\
		& \sim
		\left[\begin{matrix}4 & 0 & -4 & 0\\0 & 2 & 0 & 0\\0 & 0 & 12 & 0\end{matrix}\right] \\
		& \sim
		\left[\begin{matrix}12 & 0 & 0 & 0\\0 & 2 & 0 & 0\\0 & 0 & 12 & 0\end{matrix}\right] \\
		& \sim 
		\left[\begin{matrix}1.0 & 0 & 0 & 0\\0 & 1.0 & 0 & 0\\0 & 0 & 1.0 & 0\end{matrix}\right]
		\end{align*}
	Thus, we get $x = y = z = 0$ (the trivial solution).
	
	\newpage
	
	\exo{1.1}{4}
	\\
	We have
		\begin{align*}
		\left[\begin{matrix}3 & 1 & -2 & 3\\1 & -8 & -14 & -14\\1 & 2 & 1 & 2\end{matrix}\right]
		& \sim 
		\left[\begin{matrix}3 & 1 & -2 & 3\\0 & 25 & 40 & 45\\0 & -5 & -5 & -3\end{matrix}\right] \\
		& \sim
		\left[\begin{matrix}75 & 0 & -90 & 30\\0 & 25 & 40 & 45\\0 & 0 & 15 & 30\end{matrix}\right] \\
		& \sim
		\left[\begin{matrix}225 & 0 & 0 & 630\\0 & 75 & 0 & -105\\0 & 0 & 15 & 30\end{matrix}\right] \\
		& \sim 
		\left[\begin{matrix}1.0 & 0 & 0 & 2.8\\0 & 1.0 & 0 & -1.4\\0 & 0 & 1.0 & 2.0\end{matrix}\right]
		\end{align*}
	The solution is then $x = 2.8$, $y = -1.4$, and $z = 2.0$.
	
	\newpage
	
	\exo{1.1}{6}
	\\
	We have
		\begin{align*}
		\left[\begin{matrix}2 & 3 & 1 & 4\\1 & 9 & -4 & 2\\1 & -1 & 2 & 3\end{matrix}\right]
		& \sim 
		\left[\begin{matrix}2 & 3 & 1 & 4\\0 & -15 & 9 & 0\\0 & 5 & -3 & -2\end{matrix}\right] \\
		& \sim
		\left[\begin{matrix}10 & 0 & 14 & 20\\0 & -15 & 9 & 0\\0 & 0 & 0 & -6\end{matrix}\right]
		\end{align*}
	The last line is $0 = -6$, which is impossible. This means there is no solution to the system.
	
	\newpage
	
	\exo{1.1}{18}
	\\
	We have
	\begin{align*}
	\left[\begin{matrix}1 & 2 & -1 & a\\1 & 1 & -2 & b\\2 & 1 & -3 & c\end{matrix}\right] & \sim
	\left[\begin{matrix}1 & 2 & -1 & a\\0 & 1 & 1 & a - b\\0 & 3 & 1 & 2 a - c\end{matrix}\right]\\
	 & \sim 
	\left[\begin{matrix}1 & 2 & -1 & a\\0 & 1 & 1 & a - b\\0 & 0 & -2 & - a + 3 b - c\end{matrix}\right] \\
	& \sim 
	\left[\begin{matrix}1 & 2 & -1 & a\\0 & 1 & 1 & a - b\\0 & 0 & 1.0 & 0.5 a - 1.5 b + 0.5 c\end{matrix}\right]
	\end{align*}
So, there is no condition on $a$, $b$, $c$ because the last line is valid (line $[0\, 0 \, 1 \, \text{constant} ]$. Any number are admissible.

	\newpage
	
	\exo{1.1}{24}
	\\
	\underline{1st solution}: According to the lecture notes, there are $3$ variables and $4$ equations. There are more equations than variables. This means the system has no solution at all. \textsc{However, this is not true and there was a mistake in the lecture notes}.
	
	 \vspace*{1cm}
	 
	 \noindent\underline{2nd solution}: We have to reduce to the RREF.
	 
	 \begin{align*}
	 \left[\begin{matrix}1 & 1 & 2 & 0\\3 & -1 & -2 & 0\\2 & -2 & -4 & 0\\1 & 3 & 6 & 0\end{matrix}\right] 
	 & \sim \left[\begin{matrix}1 & 1 & 2 & 0\\0 & -4 & -8 & 0\\0 & -4 & -8 & 0\\0 & 2 & 4 & 0\end{matrix}\right] \\
	 & \sim \left[\begin{matrix}4 & 0 & 0 & 0\\0 & -4 & -8 & 0\\0 & 0 & 0 & 0\\0 & 0 & 0 & 0\end{matrix}\right] \\
	 & \sim \left[\begin{matrix}1.0 & 0 & 0 & 0\\0 & 1.0 & 2.0 & 0\\0 & 0 & 0 & 0\\0 & 0 & 0 & 0\end{matrix}\right]
	 \end{align*}
	We then conclude that $x = 0$ and $y + 2z = 0$. This is equivalent to 
		\begin{align*}
		x = 0 \quad \text{ and } \quad y = -2z .
		\end{align*}
	with $z$ a free variable.
	
	
\end{document}