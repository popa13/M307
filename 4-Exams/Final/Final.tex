\documentclass[addpoints, 12pt]{exam}%, answers]
\usepackage[utf8]{inputenc}
\usepackage[T1]{fontenc}

\usepackage{lmodern}
\usepackage{arydshln}
\usepackage[margin=2cm]{geometry}

\usepackage{enumitem}

\usepackage{amsmath, amsthm, amsfonts, amssymb}
\usepackage{graphicx}
\usepackage{tikz}
\usetikzlibrary{arrows,calc,patterns}
\usepackage{pgfplots}
\pgfplotsset{compat=newest}
\usepackage{url}
\usepackage{multicol}
\usepackage{thmtools}
\usepackage{wrapfig}

\usepackage{caption}
\usepackage{subcaption}

\usepackage{pifont}

% MATH commands
\newcommand{\bC}{\mathbb{C}}
\newcommand{\bR}{\mathbb{R}}
\newcommand{\bN}{\mathbb{N}}
\newcommand{\bZ}{\mathbb{Z}}
\newcommand{\bT}{\mathbb{T}}
\newcommand{\bD}{\mathbb{D}}

\newcommand{\cL}{\mathcal{L}}
\newcommand{\cM}{\mathcal{M}}
\newcommand{\cP}{\mathcal{P}}
\newcommand{\cH}{\mathcal{H}}
\newcommand{\cB}{\mathcal{B}}
\newcommand{\cK}{\mathcal{K}}
\newcommand{\cJ}{\mathcal{J}}
\newcommand{\cU}{\mathcal{U}}
\newcommand{\cO}{\mathcal{O}}
\newcommand{\cA}{\mathcal{A}}
\newcommand{\cC}{\mathcal{C}}
\newcommand{\cF}{\mathcal{F}}

\newcommand{\fK}{\mathfrak{K}}
\newcommand{\fM}{\mathfrak{M}}

\newcommand{\ga}{\left\langle}
\newcommand{\da}{\right\rangle}
\newcommand{\oa}{\left\lbrace}
\newcommand{\fa}{\right\rbrace}
\newcommand{\oc}{\left[}
\newcommand{\fc}{\right]}
\newcommand{\op}{\left(}
\newcommand{\fp}{\right)}

\newcommand{\ra}{\rightarrow}
\newcommand{\Ra}{\Rightarrow}

\renewcommand{\Re}{\mathrm{Re}\,}
\renewcommand{\Im}{\mathrm{Im}\,}
\newcommand{\Arg}{\mathrm{Arg}\,}
\newcommand{\Arctan}{\mathrm{Arctan}\,}
\newcommand{\sech}{\mathrm{sech}\,}
\newcommand{\csch}{\mathrm{csch}\,}
\newcommand{\Log}{\mathrm{Log}\,}
\newcommand{\cis}{\mathrm{cis}\,}

\newcommand{\ran}{\mathrm{ran}\,}
\newcommand{\bi}{\mathbf{i}}
\newcommand{\Sp}{\mathrm{span}\,}
\newcommand{\Inv}{\mathrm{Inv}\,}
\newcommand\smallO{
  \mathchoice
    {{\scriptstyle\mathcal{O}}}% \displaystyle
    {{\scriptstyle\mathcal{O}}}% \textstyle
    {{\scriptscriptstyle\mathcal{O}}}% \scriptstyle
    {\scalebox{.7}{$\scriptscriptstyle\mathcal{O}$}}%\scriptscriptstyle
  }
\newcommand{\HOL}{\mathrm{Hol}}
\newcommand{\cl}{\mathrm{clos}}
\newcommand{\ve}{\varepsilon}

\DeclareMathOperator{\dom}{dom}

%%%%%% Définitions Theorems and al.
%\declaretheoremstyle[preheadhook = {\vskip0.2cm}, mdframed = {linewidth = 2pt, backgroundcolor = yellow}]{myThmstyle}
%\declaretheoremstyle[preheadhook = {\vskip0.2cm}, postfoothook = {\vskip0.2cm}, mdframed = {linewidth = 1.5pt, backgroundcolor=green}]{myDefstyle}
%\declaretheoremstyle[bodyfont = \normalfont , spaceabove = 0.1cm , spacebelow = 0.25cm, qed = $\blacktriangle$]{myRemstyle}

%\declaretheorem[ style = myThmstyle, name=Th\'eor\`eme]{theorem}
%\declaretheorem[style =myThmstyle, name=Proposition]{proposition}
%\declaretheorem[style = myThmstyle, name = Corollaire]{corollary}
%\declaretheorem[style = myThmstyle, name = Lemme]{lemma}
%\declaretheorem[style = myThmstyle, name = Conjecture]{conjecture}

%\declaretheorem[style = myDefstyle, name = D\'efinition]{definition}

%\declaretheorem[style = myRemstyle, name = Remarque]{remark}
%\declaretheorem[style = myRemstyle, name = Remarques]{remarks}

\newtheorem{theorem}{Théorème}
\newtheorem{corollary}{Corollaire}
\newtheorem{lemma}{Lemme}
\newtheorem{proposition}{Proposition}
\newtheorem{conjecture}{Conjecture}

\theoremstyle{definition}

\newtheorem{definition}{Définition}[section]
\newtheorem{example}{Exemple}[section]
\newtheorem{remark}{\textcolor{red}{Remarque}}[section]
\newtheorem{exer}{\textbf{Exercice}}[section]


\tikzstyle{myboxT} = [draw=black, fill=black!0,line width = 1pt,
    rectangle, rounded corners = 0pt, inner sep=8pt, inner ysep=8pt]

\begin{document}
	\noindent \hrulefill \\
	\noindent MATH-307 \hfill Created by Pierre-O. Paris{\'e}\\
	Final \hfill Summer 2022\\\vspace*{-0.7cm}

\noindent\hrulefill
	
\vspace*{1cm}

\noindent\makebox[\textwidth]{\textbf{Last name:}\enspace \hrulefill}
\makebox[\textwidth]{\textbf{First name:}\enspace\hrulefill}

\vspace*{1cm}
\begin{center}
\gradetable[h][questions]
\end{center}
\vspace*{1cm}

{\bf Instructions:} Make sure to write your complete name on your copy. You must answer all the questions below and write your answers directly on the questionnaire. At the end of the 80 minutes, return your copy. 

No devices such as a smart phone, cell phone, laptop, or tablet can be used during the exam. You are not allowed to use the lecture notes or the textbook. You may bring one 2-sided cheat sheet of handwriting notes. You may use a digital calculator (no graphical calculators or symbolic calculators will be allowed).

You must show ALL your work to have full credit. An answer without justification is worth no point.

\vspace*{2cm}
\noindent May the Force be with you! \hfill Pierre-Olivier Parisé

\vfill

\noindent\textbf{Your Signature:} \hrulefill

\vspace*{1cm}

\begin{center}
\begin{minipage}{0.29\textwidth}
\begin{Huge}
\textsc{University of Hawai'i}
\end{Huge}
\end{minipage}
\begin{minipage}{0.12\textwidth}
\includegraphics[scale=0.05]{../../../../manoaseal_transparent.png}
\end{minipage}
\end{center}

\qformat{\rule{0.3\textwidth}{.4pt} \begin{large}{\textsc{Question}} \thequestion \end{large} \hspace*{0.2cm} \hrulefill \hspace*{0.1cm} \textbf{(\totalpoints\hspace*{0.1cm} pts)}}

\vspace*{0.5cm}

\newpage

\begin{questions}

\question
Verify if the following vector functions are solutions to the homogeneous system $Y' = AY$ with the given matrix $A$.
	\begin{parts}
	\part[10]
	$A = \begin{bmatrix} 0 & -1 \\ 4 & 0 \end{bmatrix}$ and $Y (x) = \begin{bmatrix} \sin (2t) + \cos (2t) \\ -2 \cos (2t) + 2 \sin (2t) \end{bmatrix}$.
	
	\part[10]
	$A = \begin{bmatrix} 0 & -3 \\ -12 & 0 \end{bmatrix}$ and $Y(x) = \begin{bmatrix} e^{6t} + e^{-6t} \\ -2e^{6t} + 2 e^{-6t} \end{bmatrix}$.
	\end{parts}
	
\newpage
	
\question
Consider the following matrix:
	\begin{align*}
	A = \begin{bmatrix}
	2 & 0 & 1 \\
	1 & 1 & 1 \\
	0 & 0 & 1
	\end{bmatrix} .
	\end{align*}
	
	\begin{parts}
	\part[5]
	Find the eigenvalues of the matrix $A$. \textit{(Hint: Use the last row to compute the determinant.)}
	
	\part[10]
	Find a basis and the dimension of each eigenspace.
	
	\part[5]
	Is $A$ diagonalizable? If so, find the diagonal matrix $D$ and the change of basis $P$ such that $D = P^{-1} A P$.
	\end{parts}	
	
\newpage
	
\question[20]
Solve the following homogeneous system of differential equations.
	\begin{align*}
	Y' = \begin{bmatrix} 0 & -1 \\ 9 & 0 \end{bmatrix} Y .
	\end{align*}
The diagonal matrix $D$ and the change of basis $P$ such that $A = P D P^{-1}$ are
	\begin{align*}
	D = \begin{bmatrix} 3i & 0 \\ 0 & -3i \end{bmatrix}
	\quad \text{ and } \quad
	P = \begin{bmatrix}
	i/3 & -i/3 \\
	1 & 1
	\end{bmatrix} .
	\end{align*}
	
	
\newpage

\question[20]
Solve the following initial value problem:
	\begin{align*}
	Y' = \left[\begin{matrix}2 & 1\\-3 & -2\end{matrix}\right] Y 
	\quad \text{ and } \quad
	Y(0) = \begin{bmatrix} 2 \\ 1 \end{bmatrix}
	\end{align*}
The diagonal matrix $D$ and the change of basis $P$ such that $A = P D P^{-1}$ are
	\begin{align*}
	D = \begin{bmatrix} 1 & 0 \\ 0 & -1 \end{bmatrix} 
	\quad \text{ and } \quad
	P = \begin{bmatrix} -1 & -1/3 \\
	1 & 1
	\end{bmatrix} .
	\end{align*}
	
\newpage

\question
Answer the following questions.

	\begin{parts}
	\part[5]
	Let $A$ be a square matrix. Suppose that $\lambda$ is an eigenvalue for $A$. Show that $\lambda^3$ is an eigenvalue for $A^3$.
	
	\part[5]
	Suppose the characteristic polynomial of a square matrix $A$ is
		\begin{align*}
		(\lambda - 2)^4 (\lambda + 2)^2 (\lambda - 1) .
		\end{align*}
	Suppose further that $\dim (E_2) = 2$, $\dim (E_{-2}) = 2$, and $\dim (E_1) = 1$. Give the Jordan Canonical Form of $A$.
	
	\end{parts}
	
\newpage

\question
Answer the following statements with \textbf{True} or \textbf{False}. Write your answer on the horizontal line at the end of each statement. Justify your answer in the white space underneath each statement.

	\begin{parts}
	\pointformat{(\hspace*{0.5cm}/ \thepoints)}
	\pointname{}
	%\pointsinrightmargin
	
	\part[2]
	For an eigenvalue $\lambda$ of a square matrix, there is only one eigenvector $v$ associated with $\lambda$.
	\begin{solution}[\stretch{1}]
	
	\end{solution}
	\answerline[False]
	
	\part[2]
	If the characteristic polynomial of a square matrix $A$ is $(\lambda - 4)^2 (\lambda + 3)^3 (\lambda + 2)^2$, then there are $12$ possible Jordan Canonical Forms of $A$.
	\begin{solution}[\stretch{1}]
	
	\end{solution}
	\answerline[False]
	
	\part[2]
	If $2$ and $3$ are the eigenvalue of a $3 \times 3$ matrix $A$ and if $\dim (E_2) + \dim (E_3) = 2$, then $A$ is diagonalizable.
	\begin{solution}[\stretch{1}]
	
	\end{solution}
	\answerline[True]
	
	\part[2]
	If $v$ is an eigenvector associated to the eigenvalue $\lambda$ and if $w$ is an eigenvector associated to the eigenvalue $\mu$, with $\mu \neq \lambda$, then $v$ and $w$ are linearly dependent.
	\begin{solution}[\stretch{1}]
	
	\end{solution}
	\answerline[False]
	
	\part[2]
	Eigenvalues and eigenvectors are important tools to solve applied scientific problems.
	\begin{solution}[\stretch{1}]
	
	\end{solution}
	\answerline[True]
	
	\end{parts}

\end{questions}


\end{document}