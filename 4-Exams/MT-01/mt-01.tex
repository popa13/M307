\documentclass[addpoints, 12pt]{exam}%, answers]
\usepackage[utf8]{inputenc}
\usepackage[T1]{fontenc}

\usepackage{lmodern}
\usepackage{arydshln}
\usepackage[margin=2cm]{geometry}

\usepackage{enumitem}

\usepackage{amsmath, amsthm, amsfonts, amssymb}
\usepackage{graphicx}
\usepackage{tikz}
\usetikzlibrary{arrows,calc,patterns}
\usepackage{pgfplots}
\pgfplotsset{compat=newest}
\usepackage{url}
\usepackage{multicol}
\usepackage{thmtools}
\usepackage{wrapfig}

\usepackage{caption}
\usepackage{subcaption}

\usepackage{pifont}

% MATH commands
\newcommand{\bC}{\mathbb{C}}
\newcommand{\bR}{\mathbb{R}}
\newcommand{\bN}{\mathbb{N}}
\newcommand{\bZ}{\mathbb{Z}}
\newcommand{\bT}{\mathbb{T}}
\newcommand{\bD}{\mathbb{D}}

\newcommand{\cL}{\mathcal{L}}
\newcommand{\cM}{\mathcal{M}}
\newcommand{\cP}{\mathcal{P}}
\newcommand{\cH}{\mathcal{H}}
\newcommand{\cB}{\mathcal{B}}
\newcommand{\cK}{\mathcal{K}}
\newcommand{\cJ}{\mathcal{J}}
\newcommand{\cU}{\mathcal{U}}
\newcommand{\cO}{\mathcal{O}}
\newcommand{\cA}{\mathcal{A}}
\newcommand{\cC}{\mathcal{C}}
\newcommand{\cF}{\mathcal{F}}

\newcommand{\fK}{\mathfrak{K}}
\newcommand{\fM}{\mathfrak{M}}

\newcommand{\ga}{\left\langle}
\newcommand{\da}{\right\rangle}
\newcommand{\oa}{\left\lbrace}
\newcommand{\fa}{\right\rbrace}
\newcommand{\oc}{\left[}
\newcommand{\fc}{\right]}
\newcommand{\op}{\left(}
\newcommand{\fp}{\right)}

\newcommand{\ra}{\rightarrow}
\newcommand{\Ra}{\Rightarrow}

\renewcommand{\Re}{\mathrm{Re}\,}
\renewcommand{\Im}{\mathrm{Im}\,}
\newcommand{\Arg}{\mathrm{Arg}\,}
\newcommand{\Arctan}{\mathrm{Arctan}\,}
\newcommand{\sech}{\mathrm{sech}\,}
\newcommand{\csch}{\mathrm{csch}\,}
\newcommand{\Log}{\mathrm{Log}\,}
\newcommand{\cis}{\mathrm{cis}\,}

\newcommand{\ran}{\mathrm{ran}\,}
\newcommand{\bi}{\mathbf{i}}
\newcommand{\Sp}{\mathrm{span}\,}
\newcommand{\Inv}{\mathrm{Inv}\,}
\newcommand\smallO{
  \mathchoice
    {{\scriptstyle\mathcal{O}}}% \displaystyle
    {{\scriptstyle\mathcal{O}}}% \textstyle
    {{\scriptscriptstyle\mathcal{O}}}% \scriptstyle
    {\scalebox{.7}{$\scriptscriptstyle\mathcal{O}$}}%\scriptscriptstyle
  }
\newcommand{\HOL}{\mathrm{Hol}}
\newcommand{\cl}{\mathrm{clos}}
\newcommand{\ve}{\varepsilon}

\DeclareMathOperator{\dom}{dom}

%%%%%% Définitions Theorems and al.
%\declaretheoremstyle[preheadhook = {\vskip0.2cm}, mdframed = {linewidth = 2pt, backgroundcolor = yellow}]{myThmstyle}
%\declaretheoremstyle[preheadhook = {\vskip0.2cm}, postfoothook = {\vskip0.2cm}, mdframed = {linewidth = 1.5pt, backgroundcolor=green}]{myDefstyle}
%\declaretheoremstyle[bodyfont = \normalfont , spaceabove = 0.1cm , spacebelow = 0.25cm, qed = $\blacktriangle$]{myRemstyle}

%\declaretheorem[ style = myThmstyle, name=Th\'eor\`eme]{theorem}
%\declaretheorem[style =myThmstyle, name=Proposition]{proposition}
%\declaretheorem[style = myThmstyle, name = Corollaire]{corollary}
%\declaretheorem[style = myThmstyle, name = Lemme]{lemma}
%\declaretheorem[style = myThmstyle, name = Conjecture]{conjecture}

%\declaretheorem[style = myDefstyle, name = D\'efinition]{definition}

%\declaretheorem[style = myRemstyle, name = Remarque]{remark}
%\declaretheorem[style = myRemstyle, name = Remarques]{remarks}

\newtheorem{theorem}{Théorème}
\newtheorem{corollary}{Corollaire}
\newtheorem{lemma}{Lemme}
\newtheorem{proposition}{Proposition}
\newtheorem{conjecture}{Conjecture}

\theoremstyle{definition}

\newtheorem{definition}{Définition}[section]
\newtheorem{example}{Exemple}[section]
\newtheorem{remark}{\textcolor{red}{Remarque}}[section]
\newtheorem{exer}{\textbf{Exercice}}[section]


\tikzstyle{myboxT} = [draw=black, fill=black!0,line width = 1pt,
    rectangle, rounded corners = 0pt, inner sep=8pt, inner ysep=8pt]

\begin{document}
	\noindent \hrulefill \\
	\noindent MATH-307 \hfill Created by Pierre-O. Paris{\'e}\\
	Midterm 01 \hfill Summer 2022\\\vspace*{-0.7cm}

\noindent\hrulefill
	
\vspace*{1cm}

\noindent\makebox[\textwidth]{\textbf{Last name:}\enspace \hrulefill}
\makebox[\textwidth]{\textbf{First name:}\enspace\hrulefill}

\vspace*{1cm}
\begin{center}
\gradetable[h][questions]
\end{center}
\vspace*{1cm}

{\bf Instructions:} Make sure to write your complete name on your copy. You must answer all the questions below and write your answers directly on the questionnaire. At the end of the 80 minutes, hand out your copy. 

No devices such as a smart phone, cell phone, laptop, or tablet can be used during the exam. You are not allowed to use the lecture notes, the textbook. You may bring one 2-sided cheat sheet of handwriting notes. You may use a digital calculator (no graphical calculator or symbolic calculator will be allowed).

You must show ALL your work to have full credit. An answer without justification worth no point.

\vspace*{2cm}
\noindent May the Force be with you! \hfill Pierre-Olivier Parisé

\vfill

\noindent\textbf{Your Signature:} \hrulefill

\vspace*{1cm}

\begin{center}
\begin{minipage}{0.3\textwidth}
\begin{Huge}
\textsc{University of Hawai'i}
\end{Huge}
\end{minipage}
\begin{minipage}{0.12\textwidth}
\includegraphics[scale=0.05]{../../../../manoaseal_transparent.png}
\end{minipage}
\end{center}

\qformat{\rule{0.3\textwidth}{.4pt} \begin{large}{\textsc{Question}} \thequestion \end{large} \hspace*{0.2cm} \hrulefill \hspace*{0.1cm} \textbf{(\totalpoints\hspace*{0.1cm} pts)}}

\vspace*{0.5cm}

\newpage

\begin{questions}

\question
Using the Gauss-Jordan Elimination Method, say if the following systems of linear equations has one solution, more than one solution, or no solution. If the system has solution(s), find the solution(s) explicitly.
	
	\begin{multicols}{2}
	\begin{parts}
	\pointformat{(\hspace*{0.5cm}/ \thepoints)}
	\pointname{}
	
	\part[10]
	$\left\{ \begin{matrix}
	2x + 3y - 4z = 3 \\
	2x + 3y - 2z = 3 \\
	4x + 6y - 2z = 7
	\end{matrix} \right.$
	
	\part[10]
	$\left\{ \begin{matrix}
	4x - 2y + 3z =0 \\
	2x + 2y - 4z = 0
	\end{matrix} \right. .$
	\end{parts}
	\end{multicols}
	
	\newpage
	
\question

Suppose we have the following system of linear equations:
	\begin{align*}
	\left\{
	\begin{matrix}
	2x - y + 3z = 42 \\
	x + y - 2z = 42 \\
	x + y + 5z = 21
	\end{matrix} \right. .
	\end{align*}
	
	\begin{parts}
	\pointformat{(\hspace*{0.5cm}/ \thepoints)}
	\pointname{}
	
	\part[5]
	Write the system in its matrix form.
	
	\part[10]
	Find the inverse of the matrix of coefficients.
	
	\part[5]
	Find the solution to the system using the inverse.
	
	\end{parts}
	
	\newpage
	
	\phantom{2}
	
	\newpage
	
\question

Suppose we have the following matrices:
	\begin{align*}
	A = \begin{bmatrix}
	-3 & 0 & 4 \\
	2 & -1 & 3 \\
	4 & 0 & 5
	\end{bmatrix}, \quad B = \begin{bmatrix}
	2 & -1 & 3 \\
	0 & 1 &0 \\
	-3 & 2 & 1
	\end{bmatrix}, 
	\quad
	C = \begin{bmatrix}
	0 & -1 & 0 & 6 \\
	0 & 0 & 0 & 5 \\
	0 & -5 & 0 & 1 \\
	3 & 4 & 0 & 3
	\end{bmatrix} .
	\end{align*}

	\begin{multicols}{2}
	\begin{parts}
	\pointformat{(\hspace*{0.5cm}/ \thepoints)}
	\pointname{}
	
	\part[5]
	Compute $2A$.
	
	\part[5]
	Compute $AB^{\top}$.
	
	\part[5]
	Compute $\det (C)$.
	
	\part[5]
	Compute $\det ( B A^\top )$.
	
	\end{parts}
	\end{multicols}
	
	\newpage
	
\question[20]

Use Cramer's rule to solve the following system of linear equations:
	\begin{align*}
	\left\{ \begin{matrix}
	3x - y = 1 \\
	y - 3z = 1 \\
	2x + z = 1
	\end{matrix} \right. .
	\end{align*}
	
\newpage
	
\question

Answer the following following

	\begin{parts}
	\pointformat{(\hspace*{0.5cm}/ \thepoints)}
	\pointname{}
	
	\part[5] 
	Suppose $A$ and $B$ are $n \times n$ symmetric matrices. Show that $(AB)^\top = BA$.
	\part[5]
	Find two matrices $A$ and $B$ such that $AB \neq BA$.
	\end{parts}
	
\newpage

\question
Answer with \textbf{True} or \textbf{False} the following statements. Write your answer on the horizontal line at the end of each statement. Justify your answer in the white space underneath the statement.

	\begin{parts}
	\pointformat{(\hspace*{0.5cm}/ \thepoints)}
	\pointname{}
	%\pointsinrightmargin
	
	\part[2]
	If $A$ is a $2 \times 2$ upper triangular matrix and $B$ is a $2 \times 2$ lower triangular matrix, then $AB$ is upper triangular.
	\begin{solution}[\stretch{1}]
	
	\end{solution}
	\answerline[False]
	
	\part[2]
	If $A$ is a $5 \times 3$ matrix and $B$ is a $5 \times 5$ matrix, then $AB$ is well-defined.
	\begin{solution}[\stretch{1}]
	
	\end{solution}
	\answerline[False]
	
	\part[2]
	If $A$ is a $n \times n$ matrix, then $A^\top A$ is a symmetric matrix.
	\begin{solution}[\stretch{1}]
	
	\end{solution}
	\answerline[True]
	
	\part[2]
	Suppose $A$ and $B$ are $n \times n$ matrices. If $A$ is invertible and $B$ is invertible, then $(AB)^{-1} = A^{-1} B^{-1}$.
	\begin{solution}[\stretch{1}]
	
	\end{solution}
	\answerline[False]
	
	\part[2]
	Prof. Parisé is surfing at Ala Moana. (No justification needed)
	\begin{solution}[\stretch{1}]
	
	\end{solution}
	\answerline[True]
	
	\end{parts}



\end{questions}

\end{document}